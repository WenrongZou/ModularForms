\section{Another Proof of Eichler-Shimura Relation}
We will proof the Eichler-Shimura Relation by using the commutative diagram, this proof is mainly come from \cite{Fry05}.
\subsection{Further discussion for Hecke operator}
As we discuss above, there is a natural map between moduli space and modular curve,
\begin{equation*}
    \psi_1: {\rm S}_1(N) \longrightarrow X_1(N), \qquad [\mathbb{C}/\Lambda_{\tau} , 1/N +\Lambda_{\tau}]\mapsto \Gamma_1(N)\tau,
\end{equation*}
to divisor groups. To make $\psi_1$ algebraic, consider the following commutative diagram
\begin{equation*}
    \begin{CD}
        {\rm S}_1(N) @>>> {\rm S}_1(1) \\
        @V\psi_1VV @VVV\\
        X_1(N) @>>> X_1(N)
    \end{CD}
\end{equation*}
given by 
\begin{equation*}
    \begin{CD}
        [\mathbb{C}/\Lambda_{\tau} , 1/N +\Lambda_{\tau}] @>>> [\mathbb{C}/\Lambda_{\tau}]\\
        @VVV @VVV\\
        \Gamma_1(N)\tau @>>> {\rm SL}_2(\mathbb{Z})\tau 
    \end{CD}
\end{equation*}
As we discuss in Hecke operator, we have the following commutative diagram
\begin{equation*}
    \begin{CD}
        {\rm Div}({\rm S}_1 (N)) @>T_p>> {\rm Div}({\rm S}_1(N)) \\
        @V\psi_1VV @VV\psi_1V \\
        {\rm Div(X_1(N))} @>T_p>> {\rm Div}(X_1(N))
    \end{CD}
\end{equation*}
The map $\psi_1: {\rm S}_1(N) \longrightarrow Y_1(N)$ from the discussion on moduli space and modular curve. Now we restrict
the vertical maps to degree-zero divisors and then the bottom row passes to Picard groups, giving a commutative diagram
\begin{equation}
    \begin{CD}
        {\rm Div}^0({\rm S}_1 (N)) @>T_p>> {\rm Div}^0({\rm S}_1(N)) \\
        @V\psi_1VV @VV\psi_1V \\
        {\rm Pic}^0(X_1(N)) @>T_p>> {\rm Pic}^0(X_1(N))
    \end{CD}
    \label{eq:hecke_div_pic}
\end{equation}

\subsection{Proof of Eichler-Shimura Relation}

Let $N$ be a positive integer and let $p\nmid N$ be prime. We will first give a description of the Hecke 
operator $T_p$ at the level of Picard groups of reduced modular curves,
\begin{equation*}
    \widetilde{T_p}:{\rm Pic}^0(\widetilde{X}_1(N)) \longrightarrow {\rm Pic}^0(\widetilde{X}_1(N)).
\end{equation*}
By Theorem \ref{theorem:reduce_Picard} the Hecke operator $\left\langle d \right\rangle $ on $X_1(N)$
reduces modulo $p$ and passes to Picard groups to give a commutative diagram 
\begin{equation}
    \begin{CD}
        {\rm Pic}^0(X_1(N)) @>\left\langle d\right\rangle _*>> {\rm Pic}^0(X_1(N)) \\
        @VVV @VVV \\
        {\rm Pic}^0(\widetilde{X}_1(N)) @>\widetilde{\left\langle d \right\rangle}_*>> {\rm Pic}^0(\widetilde{X}_1(N)) 
    \end{CD}
    \label{eq:reduced_Picard}
\end{equation}
We want a similar diagram for the Hecke operator $T_p$ on ${\rm Pic}^0(X_1(N))$, 
\begin{equation}
    \begin{CD}
        {\rm Pic}^0(X_1(N)) @>T_p>> {\rm Pic}^0(X_1(N)) \\
        @VVV @VVV \\
        {\rm Pic}^0(\widetilde{X}_1(N)) @>\widetilde{T_p}>> {\rm Pic}^0(\widetilde{X}_1(N)) 
    \end{CD}
    \label{eq:reduced_Picard2}
\end{equation}
However, since the top row here is not the pushforward of a morphism from $X_1(N)$ to $X_1(N)$, Theorem \ref{theorem:reduce_Picard}
doesn't give this diagram as it gives Equation \ref{eq:reduced_Picard}.
The way to compute the reduction $\widetilde{T_p}$ on ${\rm Pic}^0(\widetilde{X}_1(N))$ is to compute it first
in the moduli space environment. We will establish a moduli sapce diagram analogous to Equation \ref{eq:reduced_Picard2} and then use 
Igusa's Theorem to translate it to modular curves. Recall the moduli space interpretation of the Hecke operator $T_p$
on the divisor group of ${\rm S}_1(N)$,
\begin{equation}
    T_p[E,Q]=\sum_C[E/C,Q+C].
    \label{eq:T_p}
\end{equation}
The sum is taken over all order $p$ subgroups $C\subset E$ such that $C\cap \left\langle Q\right\rangle =\{0\}$,
in this case all order $p$ subgroups since $p\nmid N$.\par
Let $E$ be an elliptic curve over $\overline{\mathbb{Q}}$ with ordinary reduction at $\mathfrak{p}$ and let 
$Q\in E$ be a point of order $N$. Let $C_0$ be the kernel of the reduction map $E[p]\longrightarrow \widetilde{E}[p]$,
an order $p$ subgroup of $E$ since the ma surjects and the reduction is ordinary. 
\begin{lemma}
    For any order $p$ subgroup $C$ of $E$, 
    \begin{equation*}
        [\widetilde{E/C},\widetilde{Q+C}]=\begin{cases}
            [\widetilde{E}^{\varphi_p},\widetilde{Q}^{\varphi_p}] & \text{if } C=C_0\\
            [\widetilde{E}^{\varphi_p^{-1}},[p]\widetilde{Q}^{\varphi_p^{-1}}] & \text{if } C\neq C_0
        \end{cases}
    \end{equation*}
    \label{lm}
\end{lemma}
There are $p+1$ order $p$ subgroups $C$ of $E$, one of which is $C_0$. Define the moduli space diamond 
operator in characteristic $p$ to be 
\begin{equation*}
    \widetilde{\left\langle d\right\rangle }:\widetilde{{\rm S}}_1(N) \longrightarrow \widetilde{{\rm S}}_1(N), \qquad [E,Q]\mapsto [E,[d]Q], \quad (d,N)=1.
\end{equation*}
Then summing the formula of above Lemma over all order $p$ subgroups $C\subset E$ gives for a curve $E$ with ordinary
reduction at $\mathfrak{p}$, 
\begin{equation}
    \sum_C [\widetilde{E/C}, \widetilde{Q+C}]=(\varphi_p+p\widetilde{\left\langle p\right\rangle }\varphi_p^{-1})[\widetilde{E},\widetilde{Q}].
    \label{eq:reduced_T_p}
\end{equation}

The above Lemma extends to supersingular curves. If $E$ is an elliptic curve over $\overline{\mathbb{Q}}$ with supersingular
reduction at $\mathfrak{p}$ and $Q\in E$ is a point of order $N$, then for any order $p$ subgroup $C$ of $E$ 
\begin{equation*}
    [\widetilde{E/C},\widetilde{Q+C}]=[\widetilde{E}^{\varphi_p}, \widetilde{Q}^{\varphi_p}]=[\widetilde{E}^{\varphi_p^{-1}},[p]\widetilde{Q}^{\varphi_p^{-1}}].
\end{equation*}
Summing this over the $p+1$ such subgroups $C$ of $E$, using the first expression for $[\widetilde{E/C},\widetilde{Q+C}]$ 
once and the second expression $p$ times, shows that formula \ref{eq:reduced_T_p} applies to curves with supersingular
reduction at $\mathfrak{p}$ as well. Therefore it applies to all curves with good reduction at $\mathfrak{p}$.\par
If an elliptic curve $\widetilde{E}$ over $\overline{\mathbb{F}}_p$ has invariant $j\notin \{0,1728\}$ then the same
holds for $\widetilde{E}^{\varphi_p}$ and $\widetilde{E}^{\varphi_p^{-1}}$. Formulas \ref{eq:T_p} and \ref{eq:reduced_T_p}
combine with this observation to give a commutative diagram, where as before the primes mean to avoid some points,
only finitely many in characteristic $p$,
\begin{equation*}
    \begin{CD}
        {\rm S}_1(N)'_{\rm gd} @>T_p>> {\rm Div}({\rm S}_1(N)'_{\rm gd}) \\
        @VVV @VVV\\
        \widetilde{{\rm S}}_1(N)' @>\varphi_p+p\widetilde{\left\langle p \right\rangle }\varphi_p^{-1}>> {\rm Div}(\widetilde{{\rm S}}_1(N)')
    \end{CD}
\end{equation*}
That is, $T_p$ on ${\rm S}_1(N)'_{\rm gd}$ reduces at $\mathfrak{p}$ to $\varphi_p+p\widetilde{\left\langle p \right\rangle }\varphi_p^{-1}$.
This extends to divisor groups and then restricts to degree-0 divisors, 
\begin{equation}
    \begin{CD}
        {\rm Div}^0({\rm S}_1(N)'_{\rm gd}) @>T_p>> {\rm Div}({\rm S}_1(N)'_{\rm gd}) \\
        @VVV @VVV\\
        {\rm Div}^0(\widetilde{{\rm S}}_1(N)') @>\varphi_p+p\widetilde{\left\langle p \right\rangle }\varphi_p^{-1}>> {\rm Div}(\widetilde{{\rm S}}_1(N)')
    \end{CD}
    \label{cd:T_p_lemma}
\end{equation}
We also have a commutative diagram 
\begin{equation}
    \begin{CD}
        {\rm Div}^0(\widetilde{{\rm S}}_1(N)') @>\varphi_p+p\widetilde{\left\langle p \right\rangle }\varphi_p^{-1}>> {\rm Div}(\widetilde{{\rm S}}_1(N)') \\
        @VVV @VVV \\
        {\rm Pic}^0(\widetilde{X}_1(N)) @>\varphi_{p,*}+\widetilde{\left\langle p \right\rangle }_*\varphi_p^*>> {\rm Pic}^0(\widetilde{X}_1(N))
    \end{CD}
    \label{cd:lemma}
\end{equation}
A cube-shaped diagram now exist as follows, where all squares except possibly the back one commute.

\begin{center}
    \begin{tikzcd}[row sep=scriptsize, column sep=scriptsize]
        & {\rm Pic}^0(X_1(N)) \arrow[rr,"T_p"] \arrow[dd] & & {\rm Pic}^0(X_1(N))  \arrow[dd] \\
         {\rm Div}^0({\rm S}_1(N)'_{\rm gd}) \arrow[ur] \arrow[rr, crossing over, "T_p" near start] \arrow[dd] & & {\rm Div}^0({\rm S}_1(N)'_{\rm gd}) \arrow[ur] \\
        & {\rm Pic}^0(\widetilde{X}_1(N))  \arrow[rr,"\varphi_{p,*}+\widetilde{\left\langle p \right\rangle }_*\varphi_p^* " near start] & & {\rm Pic}^0(\widetilde{X}_1(N))  \\
        {\rm Div}^0({\rm \widetilde{S}}_1(N)') \arrow[rr,"\varphi_p+p\widetilde{\left\langle p \right\rangle }\varphi_p^{-1}"] \arrow[ur] & & {\rm Div}^0({\rm \widetilde{S}}_1(N)') \arrow[from=uu, crossing over] \arrow[ur] \\
        \end{tikzcd}
\end{center}
To establish this, note that \par
(1) The top square, a diagram in characteristic 0 relating $T_p$ on the moduli space and on the modular curve, is a 
restriction of commutative diagram \ref{eq:hecke_div_pic} \par
(2) The side squares, relating reduction at $p$ to the map from the moduli space to the modular curve, are commutative diagram
from \ref{eq:Igusa_theorem} Igusa's Theorem.\par
(3) The front square, relating $T_p$ on the moduli space to reduction at $p$, is commutative diagram \ref{cd:T_p_lemma} resulting from
Lemma \ref{lm}. \par
(4) The bottom square, relating maps in characteristic $p$ on the moduli space and on the modular curve, is commutative diagram \ref{cd:lemma}.\par
Therefore, the back square is commute. Moreover, the cube commute. We get the following result
\begin{theorem}[Eichler-Shimura Relation]
    Let $p\nmid N$. The following diagram commutes:
    \begin{equation*}
        \begin{CD}
            {\rm Pic}^0(X_1(N)) @>T_p>> {\rm Pic}^0(X_1(N)) \\
            @VVV @VVV \\
            {\rm Pic}^0(\widetilde{X}_1(N))  @>\varphi_{p,*}+\widetilde{\left\langle p \right\rangle }_*\varphi_p^*>> {\rm Pic}^0(\widetilde{X}_1(N))  \\
        \end{CD}
    \end{equation*}
    In particular since $\widetilde{\left\langle p\right\rangle } $ acts trivially on $\widetilde{X}_0(N)$, the following
    diagram commutes as well:
    \begin{equation*}
        \begin{CD}
            {\rm Pic}^0(X_0(N)) @>T_p>> {\rm Pic}^0(X_0(N)) \\
            @VVV @VVV \\
            {\rm Pic}^0(\widetilde{X}_0(N))  @>\varphi_{p,*}+\varphi_p^*>> {\rm Pic}^0(\widetilde{X}_0(N))  \\
        \end{CD}
    \end{equation*}
\end{theorem}


