\section{Shimura Taniyama conjecture}
This section is the most crucial and wonderful part of the proof of Fermat's last theorem -- Shimura Taniyama conjecture

\subsection{Introduction}
The theorem was proposed by Yutaka Taniyama in September 1955.
Here we give a brief introduction to Taniyama Yutaka, a tragic genius. Taniyama Yutaka, a Japanese mathematician, 
graduated from the University of Tokyo, his doctoral supervisor is Simura's theory of imaginary multiplication. 
In 1958 he became an assistant professor at the University of Tokyo. He tragically committed suicide before his 
marriage due to depression. A month later, the bride also committed suicide. This can't help but let us sigh 
generational genius, think he and Galois are a little short lived.
\subsection{History of the theorem}
As mentioned above, the theorem was proposed by Taniyama Yutaka 
in September 1955, but the original conjecture was so vague that many people, 
such as Weil, did not even agree with it. But at this point, another genius, Simura, and he improved on 
the rigor. Taniyama committed suicide in 1958. In the 1960s, it was linked to and was a key component of the 
Langlands Program of conjecture in unified mathematics. Conjecture was revived and popularized in the 1970s by Andre Wey, 
whose name was associated with it for a while. Despite its obvious usefulness, the depth of the problem was not appreciated until
later developments.The Taniyama-Shimura conjecture attracted quite a bit of attention in the 1980s when Gerhard 
Frey suggested that it (then a conjecture) shouldcontain Fermat's last theorem. He did this by trying to show 
that anyexample of Fermar's last Theorem would lead to an unmodular ellipticcurve. Ken Ribet later proved this result. 
In 1995, Andrew Wiles and Richard Taylor proved a special case of Taniyama-Shimura's theorem 
(thecase of semi-stable elliptic curves) that was sufficient to prove Fermar'slast Theorem.The complete 
proof was finally made in 1999 by Breuil, Conrad, Diamond,and Taylor, who built on Wiles's work and proved the 
rest piece by pieceuntil it was all done.Let's see what the theorem actually says.
\begin{theorem}[Taniyama-Shimura-Weil]
    
\end{theorem}
Let E be an elliptic curve defined over Q withconductor N, and L(E,s) =   an  
its L-function. Then  is a modular formof weight 2 and level N 
Meaning of theoremFirstly, 
it directly constitutes an important part of fermat last therom.According to this fact, 
it gives a positive answer to Fermat Last Therom.On a deeper level, 
it is actually a combination of two seeminglycompletely different worlds -- 
elliptic curves and modular forms. In other
