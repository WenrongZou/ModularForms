\documentclass[11pt]{article} 
\usepackage{amsmath} % AMS Math Package
\usepackage{amsthm} % Theorem Formatting
\usepackage{amssymb} % Math symbols such as \mathbb
\usepackage{graphicx} % Allows for eps images
\usepackage{multicol} % Allows for multiple columns
\usepackage[dvips,letterpaper,margin=1in,bottom=1in]{geometry}
\usepackage{hyperref}
\usepackage{mathrsfs}
\usepackage{amsmath,amscd}
\usepackage[all,cmtip]{xy}
\usepackage{bbm}
\usepackage{amsmath,amsthm,amssymb,mathrsfs,amsfonts,dsfont,amsxtra,amscd}
\usepackage{txfonts}
\usepackage{amssymb}
\usepackage{tikz-cd}
\usepackage{biblatex}
\addbibresource{Reference.bib}
\usetikzlibrary{cd}

\usepackage{thmtools}
\declaretheorem[numberwithin=section]{theorem}
\declaretheorem[numberwithin=section]{axiom}
\declaretheorem[numberwithin=section]{lemma}
\declaretheorem[numberwithin=section]{proposition}
\declaretheorem[numberwithin=section]{claim}
\declaretheorem[numberwithin=section]{conjecture}
\declaretheorem[sibling=theorem]{corollary}
\declaretheorem[numberwithin=section, style=definition]{definition}
\declaretheorem[numberwithin=section, style=definition]{property}
%\declaretheorem[numberwithin=section, style=definition]{problem}
\declaretheorem[numberwithin=section, style=definition]{problem}
\declaretheorem[numberwithin=section, style=definition]{example}
\declaretheorem[numberwithin=section, style=definition]{exercise}
\declaretheorem[numberwithin=section, style=definition]{observation}
\declaretheorem[numberwithin=section, style=definition]{fact}
\declaretheorem[numberwithin=section, style=definition]{construction}
\declaretheorem[numberwithin=section, style=definition]{remark}
\declaretheorem[numberwithin=section, style=remark]{question}
 

\newenvironment{solution}{%\small%
        \begin{trivlist} \item \textit{Solution}.  }{%
            \hspace*{\fill} $\blacksquare$\end{trivlist}}

%\newenvironment{proof}{%\small%
       % \begin{trivlist} \item \textit{Proof}.  }{%
             %   \hspace*{\fill} $\blacksquare$\end{trivlist}}
 \pagestyle{myheadings}

% ***********************************************************
% ********************** END HEADER *************************
% ***********************************************************

\begin{document}

\title{\bf \huge Modularity Theorem of Elliptic Curves and Eichler-Shimura Relation}
\author{Wenrong Zou, Yulin Peng \\
Advisor: Bingchen Lin
}
\date{March, 2023}
\maketitle
\newpage

\tableofcontents

\begin{abstract}
    Modular forms is a significant part of  the main theorem in this article. In the first section, we will explain 
    the definition and basic propties of modular forms, and then its simple connection with elliptic curves. 
    And the second section will explain the Hecke operators, which not only important in the theorem of modular forms 
    itself, but also select modular forms we need in the main theroem, whcih illustrate the deep connection between 
    modular forms and elliptic curves. What's more, we will discuss the differential forms on the elliptic curve. In 
    the last section, we will give two proofs of Eichler-Shimura Relation. 
\end{abstract}

    
    \section{Basic definition of modular forms}

    A modular form is a holomorphic function on the upper half complex plane $\mathcal{H} $, which have two positive integer parametes: weight $k$ and level $N$. We concentrate on level 1 first for simplicity.
    
    Let \(\mathrm{SL}_2(\mathbb{Z})\) be the group of $ 2\times 2 $ matrix with determinant 1 and integer entries.  $\forall \tau \in \mathcal{H}, \gamma \in \mathrm{SL}_2(\mathbb{Z}) $ we define $\gamma(\tau) = \frac{a\tau+b}{c\tau+d} $, where 
    $\gamma = 
        \left(
            \begin{smallmatrix}
            a & b \\
            c & d
            \end{smallmatrix}
        \right)
    $.
    A modular form of level 1 can be viewed as defined on $ \mathrm{SL}_2(\mathbb{Z})\backslash \mathcal{H} $, which means that $\forall \tau\in \mathcal{H}$ we can know the value of $\gamma(\tau), \forall \gamma$ form the value of $\tau$.

    \begin{definition}[Modular forms of level 1] \label{def1}
        A \textit{modular form of level 1 and weight k} is a holomorphic function $f: \mathcal{H} \to \mathbb{C}$ with the following propties:
        \begin{enumerate}
            \item $f(\gamma(\tau))=(c\tau +d)^kf(\tau), \forall \gamma = 
                \left(
                    \begin{smallmatrix}
                    a & b \\
                    c & d
                    \end{smallmatrix}
                \right)\in \mathrm{SL}_2(\mathbb{Z})$ and $\forall\tau \in \mathcal{H}$.
            \item $f$ is holomorphic at $\infty$.
        \end{enumerate}

        The set of all modular forms of level 1 and weight k is denoted by $M_k(\mathrm{SL}_2(\mathbb{Z}))$. 
    \end{definition}

    \begin{remark}
        \begin{enumerate}
            \item To make $f(\infty)$ well defined, we first notice that we have $f(\tau+1)=f(\tau)$ if we choose 
            $\gamma= \left(
                \begin{smallmatrix}
                1 & 1 \\
                0 & 1
                \end{smallmatrix}
            \right)$ 
            in property 1, so take the \textit{Fourier expansion} we have $$f(\tau)=\sum_{n = 0}^{\infty} a_n(f)q^n ,\quad q=e^{2\pi i\tau}.$$ So $q\rightarrow0\Leftrightarrow Im(\tau )\rightarrow\infty$, and then we define $f$ is holomorphic at $\infty$ if and only if $\lim_{Im(\tau) \to \infty}f(\tau)$ exists or equally $f(\tau)$ is bounded as $Im(\tau) \to \infty$.  
            \item If we choose $\gamma= \left(
                \begin{smallmatrix}
                0 & 1 \\
                -1 & 0
                \end{smallmatrix}
            \right)$ 
            in property 1, we have $f(\gamma(\gamma(\tau)))=(-1)^kf(\tau)$, so when $k$ is odd, $M_k(\mathrm{SL}_2(\mathbb{Z}))={0}$.
        \end{enumerate}
    \end{remark}

    And then we can define the \textit{cusp forms}:

    \begin{definition}[Cusp forms]
        A cusp form of weight k is a modular form of weight k and $a_0=0$ in its Fourier expansion or equally $\lim_{Im(\tau) \to \infty}f(\tau)=0$.

        The set of all cusp forms of level 1 and weight k is denoted by $S_k(\mathrm{SL}_2(\mathbb{Z}))$.
    \end{definition}

    The name cusp form comes from the conception of cusp point, it's $\infty$ for $\mathrm{SL}_2(\mathbb{Z})$.  

    \begin{example}\label{exp1}
        \begin{enumerate}
            \item The zero function on  $\mathcal{H}$  is a modular form of every weight, and every constant function on  $\mathcal{H}$  is a modular form of weight $0$.
            \item For nontrivial examples, let  $k>2$  be an even integer and define the Eisenstein series of weight  $k$  to be a $2$-dimensional analog of the Riemann zeta function  $\zeta(k)=\sum_{d=1}^{\infty} 1 / d^{k}$, 
            $$G_{k}(\tau)=\sum_{(c, d)^{\prime}} \frac{1}{(c \tau+d)^{k}}, \quad \tau \in \mathcal{H},$$
            where the primed summation sign means to sum over nonzero integer pairs  $(c, d) \in \mathbb{Z}^{2}-\{(0,0)\}$. We can prove that $G_{k}$  is holomorphic on  $\mathcal{H}$  and its terms may be rearranged (Ferd, p4). For any  $\gamma=\left(\begin{smallmatrix}a & b \\ c & d\end{smallmatrix}\right) \in \mathrm{SL}_{2}(\mathbb{Z})$, compute that    
            $$\begin{aligned}
            G_{k}(\gamma(\tau)) & =\sum_{\left(c^{\prime}, d^{\prime}\right)^{\prime}} \frac{1}{\left(c^{\prime}\left(\frac{a \tau+b}{c \tau+d}\right)+d^{\prime}\right)^{k}} \\
            & =(c \tau+d)^{k} \sum_{\left(c^{\prime}, d^{\prime}\right)^{\prime}}\frac{1}{\left(\left(c^{\prime} a+d^{\prime} c\right) \tau+\left(c^{\prime} b+d^{\prime} d\right)\right)^{k}}.
            \end{aligned}$$
            But as  $\left(c^{\prime}, d^{\prime}\right)$  runs through  $\mathbb{Z}^{2}-\{(0,0)\}$, so does  $\left(c^{\prime} a+d^{\prime} c, c^{\prime} b+d^{\prime} d\right)=   \left(c^{\prime}, d^{\prime}\right)\left(\begin{smallmatrix}a & b \\ c & d\end{smallmatrix}\right)$, and so the right side is  $(c \tau+d)^{k} G_{k}(\tau)$, showing that $ G_{k}$  is weakly modular of weight  $k$. Finally,  $G_{k}$  is bounded as  $\operatorname{Im}(\tau) \rightarrow \infty$, so it is a modular form.          
        \end{enumerate}
    \end{example}

    To define the modular forms for general level, we need the conception of congruence subgroup. Let $\Gamma_0(N), \Gamma_1(N), \Gamma(N)$ denote the subgroup of $\mathrm{SL}_2(\mathbb{Z})$ with all matrix of the form 
    \begin{equation*}
        \left(
        \begin{smallmatrix}
            * & * \\
            0 & *
        \end{smallmatrix}
    \right)(mod N),
    \qquad
    \left(
        \begin{smallmatrix}
            1 & * \\
            0 & 1
        \end{smallmatrix}
    \right)(mod N),
    \qquad
    \left(
        \begin{smallmatrix}
            1 & 0 \\
            0 & 1
        \end{smallmatrix}
    \right)(mod N)
    \end{equation*}
    respectively, where $*$ denotes any integer $mod N$. It's easy to see that $\Gamma(N)\subset \Gamma_1(N)\subset \Gamma_0(N)$, and when $N|M, \Gamma_i(M)\supset \Gamma_i(N), \forall i$.

    \begin{definition}[Congruence subgroups]
        A subgroup $\Gamma$ of $\mathrm{SL}_2(\mathbb{Z})$ is a \textit{congruence subgroup of level N} if it contains $\Gamma(N)$ for some $N$.
    \end{definition}

    \begin{remark}
        \begin{enumerate}
            \item $\Gamma_0(N), \Gamma_1(N), \Gamma(N)$ are congruence subgroups of level N.
            \item If $N|M$ and $\Gamma$ is a congruence subgroup of level N, then it's a congruence subgroups of level M. 
        \end{enumerate}
    \end{remark}

    A modular form of level N satisfied similar propties of Definition\ref{def1}, but we need to konw the cusp points for general congruence subgroups first. We expend  the action $\mathrm{SL}_2(\mathbb{Z})$ on $\mathcal{H}$ to $\mathcal{H}\cup \mathbb{Q}\cup {\infty}$ as the following:
    \begin{enumerate}
        \item $\forall q\in \mathbb{Q}$, we define $\gamma(q)=\frac{aq+b}{cq+d}$ is another rational number if $cq+d\neq 0$ and $\gamma(q)=\infty$ otherwise.
        \item For $\infty$, we define $\gamma(\infty)=a/c$ if $c\neq 0$ and $\gamma(\infty)=\infty$ otherwise.
    \end{enumerate}
    As $\Gamma$ is a subgroup of $\mathrm{SL}_2(\mathbb{Z})$, we can restrict this action on $\Gamma$, and the orbits of this restriction  on $\mathbb{Q}\cup {\infty}$ is the cusp points of $\Gamma$. 

    As the level 1 case, the modular form of level N need to be holomorphic at all cusp forms, to make the value of $f$ on $\mathbb{Q}$ well defined, we define an action of $\mathrm{SL}_2(\mathbb{Z})$ on holomorphic functions on $\mathcal{H}$ by: $$(f[\gamma]_k(\tau))=j(\gamma, \tau)^{-k}f(\gamma(\tau)), \quad \forall \gamma\in \mathrm{SL}_2(\mathbb{Z}) , \quad\tau \in \mathcal{H},$$ where $ j(\gamma, \tau)=c\tau+d$.

    \begin{lemma}
        For this action, $\forall \gamma, \gamma'\in \mathrm{SL}_2(\mathbb{Z})$ and $\tau\in \mathcal{H}$, we have:
        \begin{enumerate}
            \item $j(\gamma\gamma',\tau)=j(\gamma,\gamma'(\tau))j(\gamma',\tau)$,
            \item $(\gamma\gamma')(\tau)=\gamma(\gamma'(\tau))$,
            \item $[\gamma\gamma']_k=[\gamma]_k[\gamma']_k$,
            \item $Im(\gamma(\tau))=\frac{Im(\tau)}{\vert j(\gamma,\tau)\vert^2}$,
            \item $\frac{d\gamma(\tau)}{d\tau}=\frac{1}{j(\gamma,\tau)^2}$.
        \end{enumerate}
    \end{lemma}

    So this is a well defined action, it's easy to see that if $f\in M_k(\mathrm{SL}_2(\mathbb{Z}))$ then $f[\gamma]_k=f$. For a cusp point $q$ differ form $\infty$ of $\Gamma$, $\exists \gamma \in \mathrm{SL}_2(\mathbb{Z})$ s.t. $\gamma(q)=\infty$, so we can define $f(q)$ to be $f[\gamma]_k(\infty)$. With all these preparation, we can finally define the general modular forms:

    \begin{definition}[Modular forms]
        A \textit{modular form of weight k respect to $\Gamma$ (so has level N)} is a holomorphic function $f: \mathcal{H} \to \mathbb{C}$ with the following propties:
        \begin{enumerate}
            \item $f(\gamma(\tau))=(c\tau +d)^kf(\tau), \forall \gamma = 
                \left(
                    \begin{smallmatrix}
                    a & b \\
                    c & d
                    \end{smallmatrix}
                \right)\in \Gamma$ and $\forall\tau \in \mathcal{H}$, where $\Gamma$ is a congruence subgroup of level N,
            \item $f[\alpha]_k$ is holomorphic at $\infty, \forall \alpha \in \mathrm{SL}_2(\mathbb{Z})$.
            
            If in additon,
            \item $a_0=0$ in the Fourier expansion of  $f[\alpha]_k, \forall \alpha\in \mathrm{SL}_2(\mathbb{Z}) $,
        \end{enumerate}
        then $f$ is a cusp form of weight k respect to $\Gamma$.

        The set of all modular (resp. cusp) forms of weight k with respect to $\Gamma$ denoted $M_k(\Gamma)$ (resp. $S_k(\Gamma)$).  
    \end{definition}

    \begin{example}
        \begin{enumerate}
            \item All modular forms of level 1 is a modular forms of any level, so the examples in Example\ref{exp1} are  examples here too.
            \item For a generation of  Example\ref{exp1}, we extend $k$ to $k=2$, $G_{2}(\tau)$ is not a modular form, but 
            $$ G_{2,N}(\tau)=G_{2}(\tau)-G_{2}(N\tau)$$
            is a modular form of level N and weight 2 (Fred, p18).
        \end{enumerate}
    \end{example}

    We need the following quite none-trivial theorem letter, for its proof, see (Fred, p85-p91).

    \begin{theorem}[finite dimension]\label{finite dimension}
        The space $M_k(\Gamma)$ (and then $S_k(\Gamma)$) of any level as a complex vector space has finite dimension.
    \end{theorem}
    
    \section{Connection between modular forms and elliptic curves}

    In this section, we want to represent the definition domain of modular forms by the set related to elliptic curves. And then we can look modular forms as a function defined on the space of elliptic curves.
    To do this, we need the following definition first:

    \begin{definition}[Modular curve]
        For any congruence subgroup  $\Gamma$  of  $\operatorname{SL}_{2}(\mathbb{Z})$ , acting on the upper half plane  $\mathcal{H}$  from the left, the modular curve  $Y(\Gamma)$  is defined as the quotient space of orbits under  $\Gamma$ ,
        $$Y(\Gamma)=\Gamma \backslash \mathcal{H}=\{\Gamma \tau: \tau \in \mathcal{H}\}.$$
        The modular curves for  $\Gamma_{0}(N)$, $\Gamma_{1}(N)$ , and  $\Gamma(N)$  are denoted:
        $$Y_{0}(N)=\Gamma_{0}(N) \backslash \mathcal{H}, \quad Y_{1}(N)=\Gamma_{1}(N) \backslash \mathcal{H}, \quad Y(N)=\Gamma(N) \backslash \mathcal{H}.$$
    \end{definition}

    And then there is a bijection of spaces as following:

    \begin{theorem}[Moduli space and Modular curve]
        Let  $N$  be a positive integer. The moduli space for  $\Gamma_{0}(N)$  is
        $$\mathrm{S}_{0}(N)=\left\{\left[E_{\tau},\left\langle 1 / N+\Lambda_{\tau}\right\rangle\right]: \tau \in \mathcal{H}\right\} .$$
        Two points  $\left[E_{\tau},\left\langle 1 / N+\Lambda_{\tau}\right\rangle\right]$  and $ \left[E_{\tau^{\prime}},\left\langle 1 / N+\Lambda_{\tau^{\prime}}\right\rangle\right]$  are equal if and only if $ \Gamma_{0}(N) \tau=\Gamma_{0}(N) \tau^{\prime}.$  Thus there is a bijection $\psi_{0}: \mathrm{S}_{0}(N) \stackrel{\sim}{\longrightarrow} Y_{0}(N)$, $\left[\mathbb{C} / \Lambda_{\tau},\left\langle 1 / N+\Lambda_{\tau}\right\rangle\right] \mapsto \Gamma_{0}(N) \tau $.
        
        
    \end{theorem}
        
    \begin{remark}
        The element in the space $\mathrm{S}_{0}(N)$ is called the enhanced elliptic curve.
        
        Especially when $N=1$, the above gives  a bijecton between all elliptic curves and the modular space of $\mathrm{SL}_2(\mathbb{Z})$.
    \end{remark}

    And then we can define the correspondence of functions, we define the functions on  $\mathrm{S}_{0}(N)$ first. 

    \begin{definition}
        A complex-valued function  F  of the enhanced elliptic curves for  $\Gamma_{0}(N)$  is degree-$k$  homogeneous with respect to  $\Gamma_{0}(N)$  if for every nonzero complex number  $m$,
        $$F(\mathbb{C} / m \Lambda, m C) =m^{-k} F(\mathbb{C} / \Lambda, C) .$$
    \end{definition}

    The correspondence is as following:

    \begin{theorem}
        There is a bijection of degree-$k$  homogeneous function with respect to  $\Gamma_{0}(N)$ on $\mathrm{S}_{0}(N)$ and modular forms of weight k with respect to $\Gamma_{0}(N)$ by:
        $$f(\tau)=F\left(\mathbb{C} / \Lambda_{\tau},\left\langle 1 / N+\Lambda_{\tau}\right\rangle\right). $$ The modular form f is called the corresponding \textit{dehomogenized function} of $F$. 
    
    \end{theorem}

    \begin{remark}
        \begin{enumerate}
            \item Especially when $N=1$, the above gives  a bijecton between all degree-$k$  homogeneous functions with respect to  $\mathrm{SL}_2(\mathbb{Z})$ on all elliptic curves and all modular forms with level 1 and weight k.
            \item $f$  is weight-k  invariant with respect to  $\Gamma_{0}(N) $. To see this, let  $\gamma=\left(
                \begin{smallmatrix}
                a & b \\
                c & d
                \end{smallmatrix}
            \right) \in \Gamma_{0}(N)  $, and for any  $\tau \in \mathcal{H}$,  let  $m=(c \tau+d)^{-1}$ . Then, using the condition  $(c, d) \equiv(0,1)(\bmod N)$  at the third step,
            $$\begin{aligned}
            f(\gamma(\tau)) & =F\left(\mathbb{C} / \Lambda_{\gamma(\tau)}, \langle1 / N+\Lambda_{\gamma(\tau)}\rangle\right)=F\left(\mathbb{C} / m \Lambda_{\tau}, \langle m(c \tau+d) / N+m \Lambda_{\tau}\rangle\right) \\
            & =m^{-k} F\left(\mathbb{C} / \Lambda_{\tau},\langle 1 / N+\Lambda_{\tau}\rangle\right)=(c \tau+d)^{k} f(\tau) .
            \end{aligned}$$
        \end{enumerate}
    \end{remark}

    With all the preparation on the last two section, we can eventually study the mian tool to deal with modular forms: Hecke operators.

    \section{Hecke operators} 

    \subsection{$L$-series of a modular form}
 
    Let $ f$  be a cusp form of weight  $2k$  for  $\Gamma_{0}(N)$ . It can be expressed
    $$f(s)=\sum_{n \geq 1} c(n) q^{n}, \quad q=e^{2 \pi i z}, \quad c(n) \in \mathbb{C} .$$
    The  L-series of $ f$  is the Dirichlet series
    $$L(f, s)=\sum_{n \geq 1} c(n) n^{-s}, \quad s \in \mathbb{C} .$$

    There is an estimate that $\vert c(n)\vert \leq Cn^k$ for some constant $C$, so its Dirichlet series is convergent for $\mathfrak{R} > k+1$.

    The \textit{Mellin transform} can give us  an alternative definition of the $L$-series of a modular form by integral. 

    \begin{definition}[Mellin transform]
        Let $f=\sum_{n \geq 1}c(n)q^n$ be a cusp form, the Mellin transform of $f$ is:
        $$g(s)=\int_{0}^{\infty} f(iy)y^s \,\frac{dy}{y}.$$
    \end{definition}

    \begin{remark}
        Ignoring problem of convergence, we have:
        $$
        \begin{aligned}
            g(s) & =\int_{0}^{\infty} \sum_{n=1}^{\infty} c(n) e^{-2 \pi n y} y^{s} \frac{d y}{y} \\
            & =\sum_{n=1}^{\infty} c(n) \int_{0}^{\infty} e^{-t}(2 \pi n)^{-s} t^{s} \frac{d t}{t} \quad(t=2 \pi n y) \\
            & =(2 \pi)^{-s} \Gamma(s) \sum_{n=1}^{\infty} c(n) n^{-s} \\
            & =(2 \pi)^{-s} \Gamma(s) L(f, s) .
        \end{aligned}
        $$

        So this  gives an alternative definition of $L(f, s)$.
    \end{remark}

    \subsection{Connection with the $L$-series of an elliptic curves}

    By the definition, for  an elliptic curve $E$ over $\mathbb{Q}$,
    $$L(E,s)=\prod_{\text{$p$ good}}\frac{1}{1-a_pp^{-s}+p^{1-s}}\prod_{\text{$p$ bad} }\frac{1}{1-a_pp^{-s}},$$
    where $a_p$ can be defined  by $E$ and whether a prime number $p$ is good or bad can be read from the conductor $N$ of $E$.
    
    We can expand this product and get the Dirichlet series:
    $$L(E , s)=\sum_{n \geq 1}  a_nn^{-s}.$$
    It has the following propties:
    \begin{enumerate}
        \item $E \sim E'$ as elliptic curves, if and only if $L(E,s)=L(E',s)$.
        \item It has an Euler product representation, by definition.
        \item Conjecturally it can be extended analytically on the whole complex plane satisfies the functional equation 
         $$\Lambda(E,s)=\omega_E\Lambda(E,2-s),\quad\omega_E=\pm 1,$$
         where $\Lambda(E,s)=N^{s/2}(2\pi)^{-s}\Gamma(s)L(E,s)$.
    \end{enumerate}

    Our aim now is to find modular forms which  can have a $L$-function equals to a $L$-function of an  elliptic curve. We do this by finding modular forms which have $L$-functions satisfied the above three properties.  We can know from the last section that the first property can be fit automatically.

    The Hecke operators,  which will be defined on next subsection, can help us to find those modular forms satisfied the second property. We give its stretch first.

    \begin{property}
        For general $f$, $L(f,s)$ has an Euler product of the form:
        $$L(f,s)=\prod_{gcd(p,N)=1}\frac{1}{1-c(p)p^{-s}+p^{2k-1-s}}\prod_{p\vert N }\frac{1}{1-c(p)p^{-s}},$$
        if and only if:
        $$(*)
        \left\{\begin{aligned}
            c(m n) & =c(m) c(n), \quad \text { if } \operatorname{gcd}(m, n)=1 ; \\
            c(p) \cdot c\left(p^{r}\right) & =c\left(p^{r+1}\right)+p^{2 k-1} c\left(p^{r-1}\right),\quad r \geq 1, \text { if } p \text { is prime to } N \text {; } \\
            c\left(p^{r}\right) & =c(p)^{r}, r \geq 1, \quad \text { if } p \mid N .
        \end{aligned}\right.
        $$
    \end{property}
    
    This  can be proved by expand directly. And Hecke proved the following theorems.

    \begin{theorem}[Hecke]\label{Hecke1}
        The Hecke operators $T(n)$, $ n\in \mathbb{N}$ to be  defined letter have the following properties:
        \begin{enumerate}
            \item $T(m n)  =T(m)T(n), \text {if } \operatorname{gcd}(m, n)=1 ;$
            \item $T(p) \cdot T\left(p^{r}\right)  =T\left(p^{r+1}\right)+p^{2 k-1} T\left(p^{r-1}\right), r \geq 1, \text { if } p \text { is prime to } N \text {; }$
            \item $T\left(p^{r}\right) =T(p)^{r}, r \geq 1, \text {if } p \mid N;$
            \item all $T(n)$ commute.
        \end{enumerate}
    \end{theorem}
    
    \begin{theorem}[Hecke]\label{Hecke2}
        Let $f\in S_{2k}(\Gamma_0(N))$ be a simultaneously eigenvector for all $T(n)$, let $T(n)f=\lambda(n )f $ and let 
        $$f(s)=\sum_{n \geq 1} c(n) q^{n}, \quad q=e^{2 \pi i z}.$$
        Then $c(n)=\lambda(n)c(1)$.
    \end{theorem}

    \begin{remark}
        By the above theorems and properties, let $f(s)=\sum_{n \geq 1} c(n) q^{n}$  with  $c(1)=1$ be the one in the above theorem, then 
        $$L(f,s)=\prod_{gcd(p,N)=1}\frac{1}{1-c(p)p^{-s}+p^{2k-1-s}}\prod_{p\vert N }\frac{1}{1-c(p)p^{-s}}.$$
    \end{remark}

    So our aim in the following subsections is to define the Hecke operators and find the simultaneously eigenvectors.
    
    For the third property, we define an operator first:
    $$W_N:S_k(\Gamma_0(N))\to S_k(\Gamma_0(N)),\quad f\mapsto (\tau\mapsto i^kN^{-k/2}\tau^{-k}f(\frac{-1}{N\tau})).$$
    It's easily to see that $W_N^2=1$, so decomposite  by the eigenspaces we have: 
    $$S_k(\Gamma_0(N))=S_k(\Gamma_0(N))^{+1}\oplus S_k(\Gamma_0(N))^{-1}.$$

    \begin{theorem}[Hecke]
        Let $f\in S_{2k}(\Gamma_0(N))^\epsilon$, where $\epsilon=\pm 1$. Then $L(f,s)$  can be extended analytically on the whole complex plane satisfies the functional equation: 
        $$\Lambda(f,s)=\epsilon (-1)^k\Lambda(f,k-s),$$
        where $\Lambda(f ,s)=N^{s/2}(2\pi)^{-s}\Gamma(s)L(f,s)$.
    \end{theorem}

    \begin{remark}
        We know by the Mellin transform that $\Lambda(f,s)=N^{s/2}g(s)$, by some tricks we can extend the integral in $g(s)$ to the whole complex plane, so $\Lambda(f,s)$ can be extended analytically by Mellin transform.
    \end{remark}

    For $k=2$, this is exactly the functional equation that the $L$-function of an  elliptic curve has. So we archive the third aim.

    \subsection{Definition of Hecke operators}

    In Section 2, we have explained the bijection from moduli space to  modular curve, and then get a bijection of degree-k homogeneous function with respect to $\Gamma_0(N )$ on $S_0(N )$ and modular forms of weight $k$ with respect to $\Gamma_0(N )$. 
    To define Hecke operators, we concentrate on $N=1$ first. To define an operator on $S_{2k}(\Gamma_0(1))=S_{2k}(\mathrm{SL}_2(\mathbb{Z}))$, we first define a operator on  $\mathcal{L}$, the space of all elliptic curves, or equally, on $\mathcal{D}$, the free alelian group generated by $\mathcal{L}$.

    For $n\geq 1$, we define:
    $$T(n): \mathcal{D} \to \mathcal{D}, \Lambda\mapsto \sum_{(\Lambda:\Lambda')=n}\Lambda'$$
    and
    $$R(n): \mathcal{D} \to \mathcal{D}, \Lambda\mapsto n\Lambda.$$

    And we can prove by some easy properties of elliptic curves that:

    \begin{proposition}
        \begin{enumerate}
            \item $T(m n)  =T(m)\circ  T(n), \quad \text { if } \operatorname{gcd}(m, n)=1 ;$
            \item $T(p) \circ T\left(p^{r}\right)  =T\left(p^{r+1}\right)+pR(p)\circ T\left(p^{r-1}\right).$
        \end{enumerate}
    \end{proposition}

    And by using these two properties repeatedly, we can get:
    
    \begin{corollary}
        For any $m,n$,
        $$T(m)\circ T(n)=\sum_{d|gcd(m,n)} d\cdot R(d)\circ T(mn/d^2).$$
    \end{corollary}

    It's easy to see that $R(n)$ are commute to all other  operators, and by the above corollary, we have:

    \begin{corollary}
        All $T(n),R(n),n\geq 1$ are commute to each other.
    \end{corollary}

    For a function $F:\mathcal{L}\to \mathbb{C}$, it equals a function $F:\mathcal{D}\to \mathbb{C}$ by extend linearity. The action of $T(n),R(n),n\geq 1$ on it is defined by:
    $$(T\cdot F)(\Lambda)=F(T\Lambda),\quad T=T(n) \text{ or } R(n),n\geq 1.$$
    And if $F$ is degree-k homogeneous function with respect to $\Gamma_0(1)$, then by definition: $R(n)\cdot F=n^{-2k}F$. Composing this with the above properties, we have immediatly:

    \begin{proposition}\label{Hecke3}
        Let $F$ be of degree-k homogeneous with respect to $\Gamma_0(1)$, then so dose $T(n)\cdot F$, and 
        \begin{enumerate}
            \item $T(m n)\cdot F  =T(m)\cdot  T(n)\cdot F, \quad \text { if } \operatorname{gcd}(m, n)=1 ;$
            \item $T(p) \cdot T\left(p^{r}\right) \cdot F =T\left(p^{r+1}\right) \cdot F+p^{1-2k} T\left(p^{r-1}\right)\cdot F.$
        \end{enumerate}
    \end{proposition}

    And then we can finally define the Hecke operators for $N=1$.

    \begin{definition}
        Let $f\in S_{2k}(\Gamma_0(1))$, and let $F$ be the correspondence function on $\mathcal{L}$, we define Hecke operators $T(n),n\geq 1$ on $S_{2k}(\Gamma_0(N))$ to be:
        $$(T(n)\cdot f)(z)=n^{2k-1}(T(n)\cdot F)(\Lambda(z,1)),$$
        which is the correspondence function of $n^{2k-1} T(n)\cdot F$.
    \end{definition}

    For the general case $N\neq 1$, all things are similar, except in the second property of Proposition\ref{Hecke3}, where it holds only when $gcd(p,N)=1$, and when $p|N$, it is $T(p^r)\cdot F=T(p)^r \cdot F, r\geq 1$.

    The Theorem\ref{Hecke1} follows easily from Proposition\ref{Hecke3}. And for a proof of Theorem\ref{Hecke2}, see (Milne, p202-p203).

    \subsection{The spectral theorem and Petersson inner product}

    In Subsection2, we conclude that our aim in these  subsections is to define the Hecke operators and find the simultaneously eigenvectors, we have achieve the first part  in the last subsection, for the second part, we need a theorem form linear algebra first, which the proof of it can be find on (Milne, p204).

    \begin{theorem}[Spectral theorem]\label{The spectral theorem}
        Let $V$ be a finite dimension complex vector space with a positive-definite hermitian form $<,>$. Let $\alpha_1,\alpha_2,...$ be a sequence ofcommuting self-adjiont linear maps $V\to V$; then $V$ has a basis of consisting of vectors that are eigenvectors for all $\alpha_i$.  
    \end{theorem}

    So to find the  simultaneously eigenvectors, we need a hermitian form on $S_{2k}(\Gamma_0(N))$ first. We define it by a special integral, to define this integral, we need the measure defined by: $\mu (U)=\iint_U \frac{dxdy}{y^2}$ for any open set $U$ in the complex plane.
    
    \begin{proposition}
        For any open set $U$ in the complex plane, we have $\mu(\gamma U)=\mu(U), \forall \gamma \in \mathrm{SL}_2(\mathbb{R})$.
    \end{proposition}

    This can be proved by a direct computation. And then we define the hermitian form, which is called \textit{Petersson inner product}.

    \begin{definition}[Petersson]
        Let $f,g\in S_{2k}(\Gamma_0(N))$, then 
        $$<f,g>=\iint_D f\overline{g}y^{2k} \frac{dxdy}{y^2},$$
        is the Petersson inner product of $f,g$, where $D$ is the fundamental area of $\Gamma_0(N)$ (i.e. a connect set of represent elements of $Y_0(N)$).
    \end{definition}

    \begin{remark}
        We can prove easily that $f\overline{g}y^{2k}$ is invariant under $\Gamma_0(N)$, so it is well defined on $D$.
    \end{remark}

    \begin{theorem}[Petersson]\label{Petersson}
        The above integral converges provided at least one of $f$ or $g$ is a cusp form. It therefore defines a positive-definite hermitian form on the vector space $S_{2k}(\Gamma_0(N))$ of cusp forms. The Hecke operators $T(n)$ are self-adjoint for all $n$ relatively prime to $N$.
    \end{theorem}
    \begin{proof}[Proof]
        Fairly straightforward calculus, see \cite{Kna92}.
    \end{proof}

    \begin{remark}
        By composing Theorem\ref{finite dimension}, Theorem\ref{The spectral theorem}, and Theorem\ref{Petersson}, we have a decomposition:
        $$S_{2k}(\Gamma_0(N))=\oplus V_i $$
        of $S_{2k}(\Gamma_0(N))$ into a direct sum of orthogonal subspaces $V_i$, each of which is a simultaneous eigenspace for all $T(n)$ with $gcd(n, N)=1$.
    \end{remark}

    Unfortunately, $W_N$ dosen't commute with the $T(p)s,p|N$, while both of them  commute with the $T(p)s$, $gcd(p,N)=1$. To find the modular forms which satisfy this property, we need the last constraint: new forms.

    \subsection{Oldforms, Newforms and Eigenforms}

    The problem left by the last subsection has a simple remedy. If  $M \mid N$, then  $\Gamma_{0}(M) \supset \Gamma_{0}(N)$, and so  $\mathcal{S}_{2 k}\left(\Gamma_{0}(M)\right) \subset \mathcal{S}_{2 k}\left(\Gamma_{0}(N)\right) $. Recall that the  $N$  turns up in the functional equation for  $L(f, s)$, and so it is not surprising that we run into trouble when we mix  $f$s of level  $N$  with  $f$s that are really of level  $M \mid N$,  $M<N$. 
    
    The way out of the problem is to define a cusp form that is in some subspace  $\mathcal{S}_{2 k}\left(\Gamma_{0}(M)\right)$, $M \mid N$, $ M<N$, to be old. The old forms form a subspace  $\mathcal{S}_{2 k}^{\text {old }}\left(\Gamma_{0}(N)\right)$  of  $\mathcal{S}_{2 k}\left(\Gamma_{0}(N)\right)$, and the orthogonal complement  $\mathcal{S}_{2 k}^{\text {new }}\left(\Gamma_{0}(N)\right)$  is called the space of new forms. It is stable under all the operators  $T(n)$  and  $W_{N}$, and so  $\mathcal{S}_{2 k}^{\text {new }}$  decomposes into a direct sum of orthogonal subspaces  $W_{i}$, 
    $$\mathcal{S}_{2 k}^{\mathrm{new}}\left(\Gamma_{0}(N)\right)=\bigoplus W_{i}$$ 
    each of which is a simultaneous eigenspace for all  $T(n)$  with  $\operatorname{gcd}(n, N)=1$. Since the  $T(p)$  for  $p \mid N$  and  $W_{N}$  each commute with the  $T(n)$  for  $\operatorname{gcd}(n, N)=  1$, each stabilizes each  $W_{i}$. And then we have the following theorem.

    \begin{theorem}[Atkin-Lehner 1970]
        The spaces  $W_{i}$  in the above decomposition all have dimension $1$.
    \end{theorem}

    We take the modular form with $c(0)=1$ in $W_{i}$, then it is  naturally the eigenvector of  $T(p)$  for  $p \mid N$  and  $W_{N}$. So they are exactly  the modular forms we need, which is called Eigenforms.

\section{Meromorphic Differentials}
\subsection{Automorphic Forms}
Let $\hat{\mathbb{C}}$ denote the Riemann sphere $\mathbb{C}\cup \{\infty\}$. Recall that for an open subset 
$V\subset \mathbb{C}$, a function $f:V\rightarrow \hat{\mathbb{C}}$ is meromorphic if it is the zero function
or it has a Laurent series expansion truncated from the left about each point $\tau \in V$,
\begin{equation*}
    f(t)=\sum_{n=m}^{\infty} a_n (t-\tau)^n \ \text{for all $t$ in some disk about $\tau$,}
\end{equation*}
with coefficients $a_n \in \mathbb{C}$ and $a_m \neq 0$. The starting index $m$ is the order of $f$ at $\tau$, 
which also means ``order of vanishing" and denoted $\nu_{\tau} (f)$; the zero function is defined to have 
order $\nu_{\tau}(f)=\infty$. The function $f$ is holomorphic at $\tau$ when $\nu_{\tau}(f)\geqslant 0$,
it vanishes at $\tau$ when $\nu_{\tau}>0$, and it has a pole at $\tau$ whe $\nu_{\tau}(f)<0$. The set of meromorphic
functions on $V$ forms a field.\par
Let $\Gamma$ be a congruence subgroup of ${\rm SL_2} (\mathbb{Z})$. Recall the weight-$k$ operator, 
\begin{equation*}
    (f[\gamma]_k)(\tau)=j(\gamma,\tau)^{-k}f(\gamma(\tau)), \ \ j(\gamma,\tau)=c\tau +d \text{ for } \gamma=
    \begin{bmatrix}
        a & b \\
        c & d
    \end{bmatrix}.
\end{equation*}
As we mentioned before, a meromorphic function $f:\mathbb{H}\rightarrow \hat{\mathbb{C}}$ is called weakly
modular of weight $k$ with respect to $\Gamma$ if $f[\gamma]_k=f$ for all $\gamma\in \Gamma$. To discuss 
meromorphy of $f$ at $\infty$, let $h$ be the smallest positive integer such that 
$\begin{bmatrix}
    1 & h \\
    0 & 1
\end{bmatrix} \in \Gamma$. Thus $f$ has period $h$. If also $f$ has no poles in some region 
$\{\tau \in \mathbb{H}:{\rm Im}(\tau)>c\}$ then $f$ has a Laurent series on the corresponding punctured disk
about $0$,
\begin{equation*}
    f(\tau)=\sum_{n=-\infty}^{\infty} a_n q_h^n \ \text{  if ${\rm Im}(\tau)>0$, where $q_h=e^{2\pi i \tau /h}$}.
\end{equation*}
Then $f$ is meromorphic at $\infty$ if this series truncates from the left, starting at some $m\in\mathbb{Z}$
where $a_m\neq0$, or if $f=0$. The order of $f$ ar $\infty$, denoted $\nu_{\infty}(f)$, is again defined as the
starting index $m$ except when $f=0$, in which case $\nu_{\infty}(f)=\infty$. Now automorphic forms are defined
the same way as modular forms except with meromorphy in place of holomorphy.


\begin{definition}
    Let $\Gamma$ be a congruence subgroup of ${\rm SL_2}(\mathbb{Z})$ and let $k$ be an integer. A function 
    $f:\mathbb{H}\rightarrow\hat{\mathbb{C}}$ is an automorphic form of weight $k$ with respect to $\Gamma$ 
    if \par
    (1) \ $f$ is meromorphic,\par
    (2) \ $f$ is weight-$k$ invariant under $\Gamma$,\par
    (3) \ $f[\alpha]_k$ is meromorphic at $\infty$ for all $\alpha\in {\rm SL_2}(\mathbb{Z})$. \par
    The set of automorphic forms of weight $k$ with respect to $\Gamma$ is denoted $\mathcal{A}_k(\Gamma)$.
\end{definition}

Setting $k=0$ in the definition shows that $\mathcal{A}_0(\Gamma)$ is the field of meromorphic functions on 
$X(\Gamma)$, denoted $\mathbb{C}(X(\Gamma))$.

\subsection{Meromorphic Differentials}
Let $\Gamma$ be a congruence subgroup of ${\rm SL_2}(\mathbb{Z})$. The transformation rule for automorphic 
forms of weight $k$ with respect to $\Gamma$, 
\begin{equation*}
    f(\gamma(\tau))=j(\gamma,\tau)^k f(\tau), \ \gamma\in \Gamma
\end{equation*}
does not make such forms $\Gamma$-invariant. On the other hand, we have obtained that $d\gamma(\tau)=j(\gamma,\tau)^{-2} d\tau$
for $\gamma\in \Gamma$ so that at least formally the differential 
\begin{equation*}
    f(\tau)(d\tau)^{k/2}
\end{equation*} 
is truly $\Gamma$-invariant. The aim of this section is to define differentials on the Riemann surface $X(\Gamma)$. 
The first step is to define differentials locally. Let $V$ be any open subset of $\mathbb{C}$ and let $n\in\mathbb{N}$
be any natural number.
\begin{definition}
    The meromorphic differentials on $V$ of degree $n$ are 
    \begin{equation*}
        \Omega ^{\otimes n}(V)=\{f(q)(dq)^n:f \text{ is meromorphic on } V\}
    \end{equation*}
    where $q$ is the variable on $V$.
\end{definition} 
These form a vector space over $\mathbb{C}$ under the natural definitions of addition and scalar multiplication, 
$f(q)(dq)^n+g(q)(dq)^n=(f+g)(dq)^n$ and $c(f(q)(dq)^n)=(cf)(dq)^n$. The sum over all degrees, 
\begin{equation*}
    \Omega(V)=\oplus_{n\in\mathbb{N}}\Omega^{\otimes n}(V)
\end{equation*}
naturally forms a ring under the definition $(dq)^n(dq)^m=(dq)^{n+m}$

\begin{theorem}
    Let $k\in \mathbb{N}$ be even and let $\Gamma$ be a congruence subgroup of ${\rm SL_2}(\mathbb{Z})$. The map
    \begin{align*}
        w:\mathcal{A}_k(\Gamma) & \longrightarrow  \Omega^{\otimes k/2}(X(\Gamma))\\
        f & \longmapsto  (\omega_j)_{j\in J}
    \end{align*}
    where $(w_j)_{j\in J}$ pulls back to $f(\tau)(d\tau)^{k/2} \in \Omega ^{\otimes k/2}(\mathbb{H})$, is an isomorphism
    of complex vector spaces.
\end{theorem}
\begin{proof}
    The map $\omega$ is defined since we have just constructed $\omega(f)$. Clear $\omega$ is $\mathbb{C}$-linear and injective.
    And $\omega$ is surjective because every $(\omega_j)\in \Omega ^{\otimes k/2}(X(\Gamma))$ pulls back to some 
    $f(\tau)(d\tau)^{k/2}\in \Omega^{k/2}(\mathbb{H})$ with $f \in \mathcal{A}(\Gamma)$.
\end{proof}
For $k$ positive and even, $\mathcal{A}_k(\Gamma)$ takes the form $\mathbb{C(X(\Gamma))}f$ where $\mathbb{C}(X(\Gamma))$ is the 
field of meromorphic functions on $X(\Gamma)$ and $f$ is any nonzero element of $\mathcal{A}_k(\Gamma)$. 
\begin{equation*}
    \mathcal{A}_k(\Gamma)=\mathbb{C}(X(\Gamma))f=\{f_0f:f_0\in \mathbb{C}(X(\Gamma))\}
\end{equation*}
Now we focus on the specific case, the weight $2$ cusp forms $\mathcal{S}_2(\Gamma)$ are isomorphism as a complex vector
space to the degree $1$ holomorphic differentials on $X(\Gamma)$, denoted $\Omega^{1}_{hol}(X(\Gamma))$. 

\section{Elliptic Curves}
\subsection{Elliptic curves in arbitrary characteristic}
%A Weierstrass equation over $\mathbf{k}$ is any cubic equation of the form
%\begin{equation}
 %   E:y^2+a_1xy+a_3y=x^3+a_2x^2+a_4x+a_6, \ a_1,\ldots, a_6\in \mathbf{k}
%\end{equation}
%To study this, define 
%\begin{equation*}
 %   b_2=a_1^2 +4a_2, b_4=a_1a_3+2
%\end{equation*}


\begin{definition}
    Let $\bar{\mathbf{k}}$ be an algebraic closure of the field $\mathbf{k}$. When a Weierstrass equation 
    $E$ has nonzero discriminant $\Delta$ it is called nonsingular and the set 
    \begin{equation*}
        E= \{(x,y)\in \bar{\mathbf{k}}^2 \ \text{ satisfying $E(x,y)$}\} \cup \{\infty\}
    \end{equation*}
is called an elliptic curve over $\mathbf{k}$.
\end{definition}
\begin{definition}
    For any algebraic extension $\mathbf{K}/\mathbf{k}$ the set of $\mathbf{K}$-points of $E$ is a subgroup
    of $E$, 
    \begin{equation*}
        E(\mathbf{K})=\{P\in E-\{0_E\}:(x_P,y_P)\in \mathbf{K}^2\}\cup \{0_E\}
    \end{equation*}
\end{definition}
Let $N$ be a positive integer. The structure theorem for the $N$-torsion subgroup $E[N]=ker([N])$ of an elliptic is 
\begin{theorem}
    Let $E$ be an elliptic curve over $\mathbf{k}$ and let $N$ be a positive integer. Then 
    \begin{equation*}
        E[N]\cong \prod E[p^{e_p}] \ \text{where $N=\prod p^{e_p}$}
    \end{equation*}
    Also, 
    \begin{equation*}
        E[p^e]\cong (\mathbb{Z}/p^e\mathbb{Z})^2 \ \text{if $p\neq {\rm char}(\mathbf{k})$}
    \end{equation*}
    Thus $E[N] \cong (\mathbb{Z}/p^e\mathbb{Z})^2 $ if ${\rm char}(\mathbf{k})\nmid N$. On the other hand,
    if $p\nmid ={\rm char}(\mathbf{k})$, $E[p^e] \cong \mathbb{Z}/p^e\mathbb{Z}$ or $E[p^e]=\{0\}$ for all $e\geqslant1$.\par
    In particular, if ${\rm char}(\mathbf{k})=p$ then either $E[p]\cong \mathbb{Z}/p\mathbb{Z}$, in which case $E$ is called
    ordinary, or $E[p]=\{0\}$ and $E$ is supersingular. 
\end{theorem}
\begin{proposition}
    Let $E$ be a Weierstrass equation over $\mathbf{k}$. Then \\
    (a) $E$ describes an elliptic curve $\Longleftrightarrow$ \ $\Delta \neq 0$,\\
    (b) $E$ describes a curve with a node $\Longleftrightarrow$ \ $\Delta=0$ and $c_4\neq0$,\\
    (c) $E$ describes a curve with a cusp $\Longleftrightarrow$ \ $\Delta=0$ and $c_4=0$. 
\end{proposition}
\subsection{The reduction of elliptic curves over $\mathbb{Q}$}
In this part, we will discuss reduction of elliptic curves over $\mathbb{Q}$ modulo a prime $p$. Consider a general Weierstrass
equation $E$ defined over $\mathbb{Q}$, 
\begin{equation*}
    E: y^2+a_1 xy+a_3y=x^3+a_2x^2+a_4x+a_6, \ a_1,\ldots,a_6\in \mathbb{Q}
\end{equation*}
and consider admissible changes of variable over $\mathbb{Q}$. In particular the change of variable
$(x,y)=(u^2x',u^3y')$ gives a Weierstrass equation $E'$ with coefficients $a_i'=a_i/u^i$. \par
For any prime $p$ and any nonzero rational number $r$ let $\nu_p(r)$ denote the power of $p$ appearing in $r$,
i.e., $\nu _p(p^e\cdot m/n)=e\in \mathbb{Z}$ where $p\nmid mn$. Also define $\nu_p(0)=+\infty$. This is the $p$-adic
valuation, we have the following properties:
\begin{align*}
    \nu_p(rr') &=\nu_p(r)\nu_p(r') \\
    \nu_p(r+r')  &\geqslant {\rm min}\{\nu_p(r),\nu_p(r')\} \text{ with equality if $\nu_p(r)\neq\nu_p(r')$},
\end{align*}
but occurring now in the context of number fields rather than function fields. For each prime $p$ let $\nu_p(E)$
denote the smallest power of $p$ appearing in the discriminant of any integral Weierstrass equation equiavlent to 
$E$, the minimum of a set of nonnegative integers,
\begin{equation*}
    \nu_p(E)={\rm min}\{\nu_p(\Delta(E')):E' \text{integral, equivalent to $E$}\}.
\end{equation*}
\begin{definition}
    Define the global minimal discriminant of $E$ to be 
    \begin{equation*}
        \Delta_{\rm min}(E)= \prod _p p^{\nu_p(E)}.
    \end{equation*}
\end{definition}
This is a fintie product since $\nu_p(E)=0$ for all $p\nmid \Delta(E)$. $E$ is isomorphic over $\mathbb{Q}$
to an integral model $E'$ with discriminant $\Delta(E')=\Delta_{\rm min}(E)$. This is the global minimal 
Weierstrass equation $E'$, the model of $E$ to reduce modulo primes. From now on we freely assume when 
convenient that elliptic curves over $\mathbb{Q}$ are given in this form.\par
The field $\mathbb{F}_p$ of $p$ elements can be viewed as $\mathbb{Z}/p\mathbb{Z}$. That is, $\mathbb{F}_p$
is the image of a surjective homomorphism from the ring of rational integer$\mathbb{Z}$, reduction modulo $p\mathbb{Z}$,
\begin{equation*}
    \tilde{ }: \mathbb{Z}\longrightarrow \mathbb{F}_p,\quad \tilde{n}=n+p\mathbb{Z}
\end{equation*} 
This map reduces a global minimal Weierstrass equation $E$ to a Weierstrass equation $\tilde{E}$ over 
$\mathbb{F}_p$, and this defines an elliptic curve over $\mathbb{F}_p$ if and only if $p\nmid \Delta_{\rm min}(E)$.
The reduction of $E$ modulo $p$ (also called the reduction of $E$ at $p$) is 
\begin{definition}
    1. The reduction of $E$ is called \textit{good [nonsingular, stable]} if $\tilde{E}$ is again an elliptic curve, \par
    (a) \textit{ordinary } if $\tilde{E}[p]\cong p\mathbb{Z}$\par
    (b) \textit{supersingular} if $\tilde{E}=\{0\}$\\
    2. The reduction of $E$ is called \textit{bad [singular]} if $\tilde{E}$ is not an elliptic curve, in which 
    case it has only one singular point,\par
    (a) \textit{multiplicative [semistable]} if $\tilde{E}$ has a node,\par
    (b) \textit{additive [unstable]} if $\tilde{E}$ has a cusp.  
\end{definition}
Putting $E$ into global minimal form provides an almost complete description of a related integer $N_E$ called the algebraic
conductor of $E$. The global minimal discriminant and the algebraic conductor are divisible by the same primes,
\begin{equation*}
    p\nmid \Delta_{\rm min}(E) \Longleftrightarrow p\nmid N_E,
\end{equation*}
so that $E$ has good reduction at all primes $p$ not dividing $N_E$. More specifically, the algebraic conductor takes 
the form $N_E=\prod_p p^{f_p}$ where
\begin{equation*}
    f_p= \begin{cases}
    0 & \text{if $E$ has good reduction at $p$,}\\
    1 & \text{if $E$ has multiplicative reduction at $p$,}\\
    2 & \text{if $E$ has additive reduction at $p$ and $p\notin \{2,3\}$,}\\
    2+\delta_p & \text{if $E$ has additive reduction at $p$ and $p\in \{2,3\}$. }
    \end{cases}
\end{equation*}
Here $\delta_2\leqslant 6$ and $\delta_3\leqslant 3$. \par
We have already discuss the reduction of a Weierstrass equation $E$ over $\mathbb{Q}$ to $\tilde{E}$ over $\mathbb{F}_p$,
we have not yet discussed reducing the points of the curve $\tilde{E}$. This will be done in the next subsection.
\begin{definition}
    Let $E$ be an elliptic curve over $\mathbb{Q}$. Assume $E$ is in reduced form. Let $p$ be a prime and let $\tilde{E}$
    be the reduction of $E$ modulo $p$. Then 
    \begin{equation*}
        a_p(E)=p+1-|\tilde{E}(\mathbb{F}_p)|.
    \end{equation*}
\end{definition}
\begin{proposition}
     Let $E$ be an elliptic curve over $\mathbb{Q}$ and let $p$ be a prime such that $E$ has good reduction modulo 
     $p$. Let $\varphi _{p,*}$ and $\varphi _{p}^*$ be the forward and reverse maps of ${\rm Pic}^0(\tilde{E})$ 
     induced by $\varphi_p$. Then 
     \begin{equation*}
        a_p(E)= \varphi _{p,*}+\varphi _{p}^* \ \text{as endomorphisms of ${\rm Pic}^0(\tilde{E})$.   }
     \end{equation*} 
\end{proposition}

\begin{proposition}
    Let $E$ be an elliptic curve over $\mathbb{Q}$, and let $p$ be a prime such that $E$ has good reduction at $p$.
    Then the reduction is 
    \begin{equation*}
        \begin{cases}
            \text{\textit{ordinary}} &  \text{if } a_p(E)\nequiv0 ({\rm mod } \ p),\\
            \text{\textit{supersingular}} &  \text{if } a_p(E)\equiv 0 ({\rm mod} \ p).
        \end{cases}
    \end{equation*}
\end{proposition}

\subsection{Reduction of the points on Elliptic curves}
So far we have reduced a Weierstrass equation, but not the points themselves of the elliptic curve. To 
reduce the points, we first show more generally that for any positive integer $n$ the maximal ideal 
$\mathfrak{p}$ determines a reduction map 
\begin{equation*}
    \tilde{ }: \mathbf{P}^n(\overline{\mathbb{Q}})\longrightarrow \mathbf{P}^n(\overline{ \mathbb{F}_p})
\end{equation*}

\begin{proposition}
    Let $E$ be an elliptic curve over $\overline{\mathbb{Q}}$ with good reduction at $\mathfrak{p}$. Then: \par
    (a) The reduction maps on $N$-torsion, 
    \begin{equation*}
        E[N]\longrightarrow\tilde{E}[N],
    \end{equation*}
    is surjective for all $N$.\par
    (b) Any isogenous $E/C$ where $C$ is a cyclic subgroup of order $p$ also has good reduction at $\mathfrak{p}$.
    Furthermore, if $E$ has ordinary reduction at $\mathfrak{p}$ then so does $E/C$, and if $E$ has supersingular 
    reduction at $\mathfrak{p}$ then so does $E/C$.
\end{proposition}

\subsection{Reduction of algebraic curves and maps}
\begin{theorem}
    Let $C$ be a nonsingular projective algebraic curve over $\mathbb{Q}$ with good reduction at $p$.
    Then the reduction map $C\longrightarrow \widetilde{C}$ is surjective.
\end{theorem}
\begin{theorem}
    Let $C$ and $C'$ be nonsingular projective algebraic curves over $\mathbb{Q}$ with good reduction at $p$, 
    and let $C'$ have positive genus. For any morphism $h:C\longrightarrow C'$ the following diagram commutes:
    $$\begin{CD}
       C @>h>>  C' \\
       @VVV @VVV \\
       \widetilde{C} @>\tilde{h}>> \widetilde{C'}
    \end{CD}$$
    Also, ${\rm deg}(\tilde{h})={\rm deg}(h)$.
\end{theorem}

\begin{corollary}
    Let $C'$ have positive genus. Then \par
    (a) If $h:C\longrightarrow C'$ surjects then so does $\tilde{h}:\widetilde{C}\longrightarrow \widetilde{C'}$.\par
    (b) If also $k:C'\longrightarrow C''$ and $C''$ has positive genus then 
    \begin{equation*}
        \widetilde{k\circ h}=\tilde{k}\circ \tilde{h}.
    \end{equation*}
    (c) If $h$ is an isomorphism then so is $\tilde{h}$.
\end{corollary}

\begin{theorem}
    Let $C$ be a nonsingular projective algebraic curve over $\mathbb{Q}$ with good reduction at $p$. The map on 
    degree-0 divisors induced by reduction, 
    \begin{equation*}
        {\rm Div}^0(C)\longrightarrow {\rm Div}^0(\widetilde{C}), \qquad \sum n_P(P) \mapsto \sum n_P(\widetilde{P}).
    \end{equation*}
    sends principal divisors to principal divisors and therefore further induces a surjection of Picard groups,
    \begin{equation*}
        {\rm Pic}^0(C)\longrightarrow {\rm Pic}^0(\widetilde{C}), \qquad [\sum n_P(P)] \mapsto [\sum n_P(\widetilde{P})].
    \end{equation*}
    Let $C'$ also be a nonsingular projective algebraic curve over $\mathbb{Q}$ with good reduction at $p$, and let $C'$
    have positive genus. Let $h:C\longrightarrow C'$ be a morphism over $\mathbb{Q}$, and let 
    $h_*:{\rm Pic}^0(C)\longrightarrow {\rm Pic}^0(C')$ and $\tilde{h}_*:{\rm Pic}^0(\widetilde{C})\longrightarrow {\rm Pic}(\widetilde{C'})$
    be the induced forward maps of $h$ and $h$. Then the following diagram commutes: 
    $$\begin{CD}
        {\rm Pic}^0 (C) @>h_*>> {\rm Pic}^0(C')\\
        @VVV @VVV\\
        {\rm Pic}^0 (\widetilde{C}) @>\tilde{h}_*>> {\rm Pic}^0(\widetilde{C'})
    \end{CD}$$
    \label{theorem:reduce_Picard}
\end{theorem}

\begin{theorem}
    Let 
    \begin{equation*}
        \varphi: E\longrightarrow E'
    \end{equation*}
    be an isogeny over $\overline{\mathbb{Q}}$ of elliptic curves over $\overline{\mathbb{Q}}$. Then there is a reduction
    \begin{equation*}
        \tilde{\varphi}: \widetilde{E}\longrightarrow \widetilde{E'}
    \end{equation*} 
    with the properties\par
    (a) $\tilde{\varphi}$ is an isogeny.\par
    (b) If $\psi:E'\longrightarrow E''$ is also an isogeny then $\widetilde{\psi \circ \varphi}=\tilde{\psi}\circ\tilde{\varphi}$.\par
    (c) The following diagram commutes:
    $$\begin{CD}
        E @>\varphi>> E'\\
        @VVV @VVV\\
        \widetilde{E} @>\tilde{\varphi}>> \widetilde{E'}
    \end{CD}$$\par
    (d) ${\rm deg}(\tilde{\varphi})={\rm deg}(\varphi)$
\end{theorem}

\subsection{Igusa's Theorem}

Let $\mathfrak{p}$ be a maximal ideal of $\overline{\mathbb{Q}}$ lying over $p$. An elliptic curve $E$ over $\overline{\mathbb{Q}}$
with good reduction at $\mathfrak{p}$ has $j(E)\in \overline{\mathbb{Z}}_{(\mathfrak{p})}$, so this reduces at $\mathfrak{p}$
to $\widetilde{j(E)}$ in $\overline{\mathbb{F}}_p$. We avoid $j=0,1728$ in the image. Denote this with a prime in the notation
i.e., the suitable restriction of the moduli space ${\rm S}_1(N)$ over $\mathbb{Q}$ is 
\begin{equation*}
    {\rm S}_1(N)'_{\rm gd}=\{[E,Q]\in {\rm S}_1(N): E \text{has good reduction at $\mathfrak{p}$, } \widetilde{j(E)}\notin\{0,1728\} \}
\end{equation*}

In characteristic $p$, let $\widetilde{{\rm S}}_1(N)$ denote the moduli space over $\overline{\mathbb{F}}_p$, i,e., 
it consists of equivalence classes $[E,Q]$ where $E$ is an elliptic curve over $\overline{\mathbb{F}}_p$ and $Q\in E$ is 
a point of order $N$ and the equivalence relatio is isomorphism over $\overline{\mathbb{F}}_p$. Again avoid $j=0,1728$
by defining 
\begin{equation*}
    \widetilde{{\rm S}}(N)'=\{[E,Q]\in \widetilde{{\rm S}}_1(N): j(E)\notin \{0,1728\} \}
\end{equation*}
The resulting reduction map is 
\begin{equation*}
    {\rm S}_1(N)'_{\rm gd} \longrightarrow \widetilde{{\rm S}}(N)', \qquad [E_j,Q] \mapsto [\widetilde{E}_j,\widetilde{Q}].
\end{equation*}
This is a surjection.\par
Let $\sigma _{1,N}\in \mathbb{F}_p(j)[x]$ be the minimal polynomial of $x$-coordinate $x(Q)$. Define a field
\begin{equation*}
    \mathbf{K}_1(N)=\mathbb{F}_p(j)[x]/ \left\langle \sigma_{1,N}\right\rangle 
\end{equation*}
We have the result that $\mathbf{K}_1(N)\cap \overline{\mathbb{F}}_p=\mathbb{F}_p$, so $\mathbf{K}_1(N)$ is a function field over 
$\mathbb{F}_p$.
\begin{theorem}[Igusa' Theorem]
    Let $N$ be a positive integer and let $p$ be a prime with $p\nmid N$. The modular curve $X_1(N)$ has good reduction at $p$.
    There is an isomorphism of functions fields 
    \begin{equation*}
        \mathbb{F}_p(\widetilde{X}_1(N))\longrightarrow \mathbf{K}_1(N).
    \end{equation*}
    Moreover, reducing the modular curve is compatible with reducing the moduli space in that the following diagram commutes:
    $$\begin{CD}
        {\rm S}_1(N)'_{\rm gd} @>\psi_1>> X_1(N)\\
        @VVV @VVV \\
        \widetilde{{\rm S}}_1(N)' @>\tilde{\psi_1}>> \widetilde{X}_1(N).
    \end{CD}$$
    Here the top row is the map $[E_j,Q]\mapsto (j,x(Q))$ to the planar model followed by the birational equivalence to $X_1(N)$,
    and similarly for the bottom row but in characteristic $p$.
\end{theorem}
The diagram extends to divisor groups, restricts to degree $0$ divisors, and takes principal divisors to principal divisors, giving
a modified diagram as below:
\begin{equation}
    \begin{CD}
        {\rm Div}^0({\rm S}_1(N)'_{\rm gd}) @>>> {\rm Pic}^0(X_1(N))\\
        @VVV @VVV\\
        {\rm Div}^0({\widetilde{{\rm S}}}(N)') @>>> {\rm Pic}^0(\widetilde{X}_1(N))
    \end{CD}
    \label{eq:Igusa_theorem} 
\end{equation}






\section{Statement of the main theorem}
\begin{theorem}[Eichler-Shimura]
    Let $f=\sum c(n) q^n$ be a normalized newform on $\Gamma_0(N)$. If all $c(n)\in \mathbb{Z}$, then exist an elliptic curve $E_f$
     of conductor $N$ such that $L(E_f,s)=L(f,s)$ 
    
\end{theorem}

First, we want to know how to get a elliptic curve from a cusp form, then we consider the relation between L-function of cusp form 
 and elliptic curve.
\section{Get an Elliptic curve from a cusp form}
A differential one-form on an open subset of $\mathbb{C}$ is simply an expression $\omega =fdz$, with $f$ a meromorphic function. Given a smooth
Given a smooth curve $\gamma $
\begin{equation*}
    t\mapsto z(t):[a,b]\rightarrow \mathbb{C}, \ [a,b]=\{t\in \mathbb{R}| a\leqslant t \leqslant b\}
\end{equation*}
we can form the integral
$\int_{\gamma} \omega =\int_{a}^{b} f(z(t))\cdot z'(t)\cdot dt \in \mathbb{C}$ 
Now consider a compact Riemann surface $X$. If $\omega$ is a differential one-form on $x$ and $(U_i,z_i)$ is a coordinate neighbourhood for $X$, then 
 $\omega |U_i=f_i(z_i)d{z_i}$. If $(U_j,z_j)$ is a second coordinate neighbourhood, so that $z_j=w(z_j)$ on $U_i \cap U_j$, then 

 \begin{equation*}
    f_i(z_i)dz_i=f_j(w(z_i))w'(z_i)dz_i
 \end{equation*}
 on  $U_i \cap U_j$. Thus, to give a differential one-form on $X$ is to give differential one-forms $f_idz_i$ on each $U_i$, satisfying the above equation
 on the overlaps. For any (real) curve $\gamma: I \rightarrow X$ and differential one-form $\omega$ on $X$, the integral $\int_{\gamma}\omega$
 make sense.
\begin{definition}
    A differential one-form is holomorphic if it is represented on the coordinated neighbourhoods by forms $fdz$ with $f$ holomorphic.
\end{definition} 
\subsection{Jacobian variety of a Riemann surface}
Consider a Riemann surface with genus 1 and a nonzero holomorphic differential one-form $\omega$. We choose a point $P_0\in X$ and try to 
define a map 
\begin{equation*}
    P\mapsto \int_{P_o}^{P}\omega : X\rightarrow \mathbb{C}
\end{equation*}
This is not well-defined because the value of the integral depends on the path we choose from $P_0$ to $P$ $--$ nonhomotopic paths
may give different answers. From the result in the Algebraic Topology, if we choose a basis $\{\gamma_1,\gamma_2\}$ for $H_1(X,\mathbb{Z})$
(equivalently, a basis for $\pi_1(X,P_0)$), then the integral is well-defined modulo the lattice $\Lambda$ in $\mathbb{C}$ generated by 
\begin{equation*}
    \int_{\gamma_1} \omega, \int_{\gamma_2}\omega.
\end{equation*}
In this way, we obtain an isomorphism 
\begin{equation*}
    P\mapsto \int_{P_0}^{P} \omega:X\rightarrow \mathbb{C}/\Lambda
\end{equation*}
Jacobi and Abel made a similar construction for any compact Riemann surface $X$. It is an important fact that the holomorphic differential
one-form on a Riemann surface of genus $g$ form a complex vector space $\Omega^1(X)$ of dimension $g$. Now we suppose that $X$ is 
a Riemann sufrce with genus $g$, and let $\omega_1,\ldots ,\omega_g$ be a basis for the vector space $\Omega^1(X)$ of holomorphic one-forms
on $X$. Choose a point $P_0\in X$. Viewing $X$ as a sphere with $g$ handles where $g$ is the genus of $X$, let 
$A_1,\ldots,A_g$ be longitudinal loops around each handle like arm-bands, and let $B_1,\ldots,B_g$ be latitudinal loops arod each handle like
equators.Then consider the (first) homology group of $X$,
\begin{equation*}
    H_1(X,\mathbb{Z})=\mathbb{Z}\int_{A_1}\oplus\cdots \oplus \mathbb{Z}\int_{A_g}\oplus \mathbb{Z}\int_{B_1}\oplus\cdots \oplus \mathbb{Z}\int_{B_g}
    \cong \mathbb{Z}^{2g}
\end{equation*}
Similarily, there is a map
\begin{equation*}
    P\mapsto (\int_{P_0}^{P}\omega_1,\ldots,\int_{P_0}^{P}\omega_g):X\rightarrow \mathbb{C}^g/\Lambda
\end{equation*}
is well-defined. The quotient $\mathbb{C}^g/\Lambda$ is a comples manifold, called the jacobian variety ${\rm Jac}(X)$ of $X$, which
can be considered to be a higher-dimensional analogue of $\mathbb{C}/\Lambda$. Note that it is a commutative group.\par
We can define ${\rm Jac}(X)$ more canonically. Let $\Omega^1(X)^{\spcheck}$ be the dual of $\Omega^1(X)$ as a complex vector space.
For any $\gamma\in H_1(X,\mathbb{Z})$, 
\begin{equation*}
    \omega\mapsto\int_{\gamma}^{} \omega
\end{equation*}
is an element of $\Omega^1(X)^{\spcheck}$, and in this way we obtain an injective homomorphism
\begin{equation*}
    H_1(X,\mathbb{Z})\hookrightarrow \Omega^1(X)^{\spcheck}
\end{equation*}
which identifies $H_1(X,\mathbb{Z})$ with a lattice in $\Omega^1(X)^{\spcheck}$. Define 
\begin{equation*}
    {\rm Jac}(X)=\Omega^1(X)^{\spcheck}/H_1(X,\mathbb{Z}).
\end{equation*}
When we fix a point $P_0\in X$, any $P\in X$ defines an element $F_P$ of ${\rm Jac}(X)$,
\begin{equation*}
    F_P:\omega\mapsto \int_{P_0}^{P}\omega \ \ mod \ H_1(X,\mathbb{Z})
\end{equation*}
and so we get a map $X\rightarrow {\rm Jac}(X)$. The choice of a different $P_0$ gives a map that differs from the first only by a translation.\par

Now we apply above theory to the Riemann surface $X_0(N)$. We have known that $X_0(N)$ is a Riemann surface with genus 1. 
\begin{proposition}
    Let $\pi$ be the quotient map $\mathbb{H}^*\rightarrow\Gamma_0(N)\setminus \mathbb{H}^*$, and for any holomorphic differential 
    $\omega$ on $\Gamma_0(N)\setminus \mathbb{H}^*$, set $\pi^*\omega=fdz$. Then $\omega\mapsto f $ is an isomorphism from the space of holomorphic
    differentials on $\Gamma_0(N)\setminus \mathbb{H}^*$ to $\mathcal{S} _2 (\Gamma_0(N))$.
\end{proposition}

By the above Proposition we get an isomorphism $\mathcal{S_2} (\Gamma _0(N)) \cong \Omega^1 (X_0(N))$. 
The Hecke operator $T(n)$ acts on $\mathcal{S}_2(\Gamma_0(N))$, and hence on the vector space $\Omega^1(X_0(N))$ and its dual.
Moreover, the Hecke operator acts on ${\rm Jac}(X)$.
\begin{proposition}
    There is a canonical action of $T(n)$ on $H_1(X_0(N),\mathbb{Z})$, which is compatible with the map 
    $H_1(X_0(N),\mathbb{Z})\rightarrow \Omega^1(X_0(N))^{\spcheck}$. In other words, the action of $T(n)$ on $\Omega^1(X_0(N))^{\spcheck}$
    stabilizes its sublattice $H_1(X_0(N),\mathbb{Z})$, and therefore induces an action on the quetient ${\rm Jac}(X_0(N))$.
\end{proposition}
Now let $f=\sum c(n) q^n$ be a normalized newform for $\Gamma_0(N)$ with $c(n)\in \mathbb{Z}$. The map 
\begin{equation*}
    \alpha \mapsto \alpha(f):\Omega^1(X_0(N))^{\spcheck}\rightarrow \mathbb{C}
\end{equation*}
identifies $\mathbb{C}$ with the largest quotient of $\Omega^1(X_0(N))^{\spcheck}$ on which each $T(n)$ acts as multiplication
by $c(n)$. The image of $H_1(X_0(N,\mathbb{Z}))$ is a lattice $\Lambda_f$, and we set $E_f=\mathbb{C}/\Lambda_f$ $--$ it is an elliptic
curve over $\mathbb{C}$. Note that have constructed maps
\begin{equation*}
    X_0(N)\rightarrow {\rm Jac}(X_0(N))\rightarrow E_f
\end{equation*}
Moreover, the inverse image of the differential on $E_f$ represented by $dz$ is the differential on $X_0(N)$ represented by $fdz$.

\begin{theorem}
    Let $f=\sum c(n)q^n$ be a newform in $\mathcal{S} (\Gamma_0(N))$, normalized to have $c(1)=1$, and assume that all $c(n)\in \mathbb{Z}$.
    Then there exists an elliptic curve $E_f$ and a map $\alpha:X_0(N)\rightarrow E_f$ with the following properties:\par
    (a) \ $\alpha$ factors uniquely through ${\rm Jac}(X_0(N)),$
    \begin{equation*}
        X_0(N)\rightarrow {\rm Jac}(X_0(N)) \rightarrow E_f,
    \end{equation*}  
    and the second map realizes $E_f$ as the largest quotient of ${\rm Jac}(X_0(N))$ on which the endomorphism $T(n)$ and $c(n)$ of 
    ${\rm Jac}(X_0(N))$ agree.\par
    (b) \ The inverse image of an invariant differential $\omega$ on $E_f$ under $\mathbb{H}\rightarrow X_0(N) \rightarrow E_f$ is 
    a nonzero rational multiple of $fdz$.
    \label{Thm:Modularity}

\end{theorem}

\section{The relation between the L-Series of $E_f$ and the L-Series of $f$ }

\subsection{The Hecke correspondence}
\begin{definition}
    Let $X$ and $X'$ be projective nonsingular curves over a algebraically closed field $k$. A correspondence $T$ between $X$ and $X'$ 
    is a pair of finite surjective regular maps, denoted $T:X \vdash X' $
    \begin{equation*}
        X \stackrel{\alpha}{\longleftarrow} Y \stackrel{\beta}{\longrightarrow}X'
    \end{equation*}
\end{definition}

A correspondence $T$ can be viewed as a many-valued map $X\rightarrow X'$ sending a point $P\in X(k)$ to the set $\{\beta(Q_i)|Q_i\in \alpha^{-1}(P)\}$
As we defined before, $\rm{Div} (X)$ is the free Abelian group on the points of $X$, so the element of $\rm{Div} (X)$ is a finite sum 
\begin{equation*}
    D=\sum n_P [P], \ n_p\in \mathbb{Z}, \ P\in X(k)
\end{equation*} 
Therefore, a correspondence $T$ then induces a map between $T:\rm{Div} (X) \rightarrow \rm{Div} (X')$
\begin{equation*}
    [P] \mapsto \sum_i [\beta(Q_i)]
\end{equation*}
This map multiplies the degree of a divisor by $\rm{deg} (\alpha)$. It therefore sends the divisors of degree $0$ in $\rm{Div} (X)$
to the divisors of degree $0$ in $\rm{Div} (X')$ and sends the principal divisors of $\rm{Div} (X)$
to the principal divisors of $\rm{Div} (X')$. Define $J(X):=\rm{Div}^{0} (X)/{principal divisors}$. Then a correspondence $T$ induces
a map $T:J(X)\rightarrow J(X')$.
\begin{definition}
    Let $T$ is a correspondence 
    \begin{equation*}
        X \stackrel{\alpha}{\longleftarrow} Y \stackrel{\beta}{\longrightarrow}X
    \end{equation*}
    then the transpose $T^{tr}$ of $T$ is the correspondence 
    \begin{equation*}
        X \stackrel{\beta}{\longleftarrow} Y \stackrel{\alpha}{\longrightarrow}X
    \end{equation*}
\end{definition}
Let $T$ is a correspondence $T:X\vdash X'$ 
\begin{equation*}
        X \stackrel{\alpha}{\longleftarrow} Y \stackrel{\beta}{\longrightarrow}X'
\end{equation*}
For any regular function $f$ on $X'$, we define $T(f)$ to be the regular function on $X$, $T(f): P\mapsto \sum f(\beta Q_i)$.
So $T$ induces a homomorphism $\Omega^{1}\rightarrow\Omega^{1}(X)$.
Now we use above theory on the modular curve $Y_0(N)$. 
\begin{definition}
    Let $p \nmid N$, the Hecke correspondence $T(p):Y_0(N)\rightarrow Y_0(N)$ is defined below
    \begin{equation*}
        Y_0(N)\stackrel{\alpha}{\longleftarrow}Y_0(pN) \stackrel{\beta}{\longrightarrow} Y_(N)
    \end{equation*}
where $\alpha$ is the projection map and $\beta$ is the map induced by $z\mapsto pz:\mathbb{H}\rightarrow\mathbb{H}$
\end{definition}
First, we consider the point in $Y_0(N)$. As we discuss before, a point of $Y_0(pN)$ is represented by a pair of $(E,S)$ where $E$
 is an elliptic curve and $S$ is a cyclic subgroup of $E$ of order $pN$. Because $p\nmid N$, any subgroup $S$ of order $pN$ can be 
 decomposes uniquely into subgroups of $N$ and $p$, $S=S_N \times S_p$. The map $\alpha$ sends the point represented by $(E,S)$ 
 to the point represented by $(E,S_N)$ and $\beta$ sends $(E,S)$ to the point represented by $(E/S_p,S/S_p)$. Recall that $E_p$ 
 has $p+1$ cyclic subgroups, the correspondence is $1:p+1$.\par
 The unique extension of $T(p)$ to a correspondence $X_0(N)\rightarrow X_0(N)$ acts on $\Omega^{1}(X_0(N))=\mathcal{S}_2 (\Gamma_0(N))$

 \subsection{The Frobenius map}
 Let $C$ be a curve defined over the algebraic closure $\mathbb{F}^{al}$ of $\mathbb{F}_p$. If $C$ defined by equations 
 \begin{equation*}
    \sum a_{i_0i_1\cdots}X_0^{i_0}X_1^{i_1}\cdots=0
 \end{equation*}

 then we let $C^{(p)}$ be the curve defined by the equations 
 \begin{equation*}
    \sum a_{i_0i_1\cdots}^p X_0^{i_0}X_1^{i_1}\cdots=0
 \end{equation*}
 and we let the Frobenius map $\varphi: C\rightarrow C{(p)}$ send the point ${(b_0:b_1:b_2:\cdots)}$ to ${(b_0^p:b_1^p:b_2^p:\cdots)}$.
 If $C$ is defined over $\mathbb{F}_p$, then $C=C^{(p)}$ and $\varphi_p$ is the Frobenius map defined earlier.\par
 Recall a result from the field extension.
 \begin{proposition}
    Let $K$ be a field extension of $F$. If $S$ and $I$ are the separable and purely inseparable closures of $F$ in $K$, respectively,
    then $S$ and $I$ are field extension of $F$ with $S/F$ separable, $I/F$ purely inseparable, and $S\cup I=F$. If $K/F$ s algebraic,
    then $K/F$ is purely inseparable.
 \end{proposition}

 Recall that a nonconstant morphism $\alpha: C\rightarrow C'$ of curves defines an inclusion $\alpha^*: k(C')\hookrightarrow k(C)$
 of function fields, and that the degree of $\alpha$ is defined to be $[k(C):\alpha^*k(C')]$. The map $\alpha$ is said to be separable
 or purely inseparable according as $k(C)$ is a separable or purely inseparable extension of $\alpha^*k(C')$. If the separable degree 
 of $k(C)$ over $\alpha^*k(C')$ is $m$, then the map $C(k^{al})\rightarrow C'(k^{al})$ is $m:1$, except over the finite set where 
 it is ramified.
 \begin{proposition}
    The Frobenius map $\varphi_p: C\rightarrow C^{(p)}$ is purely inseparable of degree $p$, and any purely inseparable map 
    $\varphi: C\rightarrow C'$ of degree $p$ (of complete nonsingular curves) factors as 
    \begin{equation*}
        C\stackrel{\varphi_p}{\longrightarrow}C^{(p)}\stackrel{\approx }{\longrightarrow} C'
    \end{equation*}
    \label{Thm:Frobenius_map}
 \end{proposition}

 \subsection{The Eichler-Shimura relation}
 For almost all primes $p\nmid N$, $X_0(N)$ will reduce to a nonsingular curve $\tilde{X}_0(N)$. In fact, it is known that $X_0(N)$
 has a good reduction for all primes $p\nmid N$, but this is hard to prove. It is easy to see that $X_0(N)$ does not have good reduction
  at primes dividing $N$.
 \begin{theorem}
    For a prime $p$ where $X_0(N)$ has good reduction,
    \begin{equation*}
        \tilde{T}(p)=\varphi_p + \varphi_p^{tr}
    \end{equation*}
 \end{theorem}
\begin{proof}
    We sketch a proof that they agree as many-valued maps on an open subset of $\tilde{X_0}(N)$. We will state another proof
    by using the commutative diagram. \par
    As we discuss above, there is a isomorphism between $\psi_0: {\rm S_0}(N)\stackrel{\sim }{\longrightarrow} Y_0(N)$, where 
    \begin{equation*}
        {\rm S_0}(N)=\{\text{enhanced elliptic curves for $\Gamma_0(N)$}\}/\sim
    \end{equation*} 
    Over $\mathbb{Q}_p^{\rm al}$ we have the following description of $T(p)$: a point $P$ on $Y_0(N)$ is represented by a 
    homomorphism of elliptic curves $\alpha:E\rightarrow E'$ with cyclic kernel of order $N$; let $S_0,\ldots,S_p$ be the 
    subgroups of order $p$ in $E$; then $T_p(P)=\{Q_0,\ldots ,Q_p\}$ where $Q_i$ is represented by $E/S_i \rightarrow E'/\alpha(S_i)$.\par
    Consider a point $\tilde{P}$ on $\tilde{X_0}(N)$ with coordinates in $\mathbb{F}$ - by Hensel's lemma it will lift to a 
    point on $X_0(N)$ with coordinates in $\mathbb{Q}_p^{\rm al}$. Ignoring a finite number of points of $\tilde{X_0}(N)$,
     we can suppose $\tilde{P}\in \tilde{Y}_0(N)$ and hence is represented by a map $\tilde{\alpha}:\tilde{E}\rightarrow \tilde{E'}$
     where $\alpha: E\rightarrow E'$ has cyclic kernel of order $N$. By ignoring a further finite number of points, we 
     may suppose that $\tilde{E}$ has $p$ points of order dividing $p$.\par
     Let $\alpha: E\rightarrow E'$ be a lifting of $\tilde{\alpha}$ to $\mathbb{Q}_p^{\rm al}$. The reduction map 
     $E_p(\mathbb{Q}_p^{\rm al})\rightarrow \tilde{E_p}(\mathbb{F}_p^{\rm al})$ has a kernel of order $p$. Renumber the 
     subgroups of order $p$ in $E$ so that $S_0$ is the kernel of this map. Then each $S_i, i\neq 0$, maps to subgroup of 
     order $p$ in $\tilde{E}$.\par
     Then map $\tilde{T}(p): \tilde{E}\rightarrow \tilde{E}$ has factorizations
     \begin{equation*}
        \tilde{E}\stackrel{\varphi}{\longrightarrow} \tilde{E}/S_i \stackrel{\psi}{\longrightarrow} \tilde{E}, \ i=0,1, \ldots , p.
     \end{equation*}
     When $i=0$, $\varphi$ is a purely inseparable map of degree $p$ (it is the reduction of the map $E\rightarrow E/S_0$
     - it therefore has degree $p$ and has zero kernel), and so $\psi$ must be separable of degree $p$ (we are assuming 
     $\tilde{E}$ has $p$ points of order dividing $p$). By Proposition \ref{Thm:Frobenius_map} shows that there is an isomorphism 
     $\tilde{E}^{(p)} \rightarrow \tilde{E}/S_0$. Similarily $\tilde{E'}^{(p)}\approx \tilde{E'}/S_0$. Therefore $Q_0$ is represented
     by $\tilde{E}^{(p)}\rightarrow\tilde{E'}^{(p)}$, which also represents $\varphi _p(P)$. \par
     When $i\neq 0$, $\varphi$ is separable (its kernel is the reduction of $S_i$), and so $\psi$ is purely inseparable.
     Therefore $\tilde{E}\approx \tilde{E}_i^{(p)}$, and similarly $\tilde{E'}\approx \tilde{E'_i}^{(p)}$, where 
     $\tilde{E_i}=\tilde{E}/S_i$ and $\tilde{E'_i}=\tilde{E'}/S_i$ It follows that $\{Q_1,\ldots, Q_p\}=\varphi^{-1}(P)=\varphi^{\rm tr}(P)$.

\end{proof}


 \subsection{The zeta function of an elliptic curve}
For an Elliptic curve $E=\mathbb{C}/\Lambda$ over $\mathbb{C}$, the degree of a nonzero endomorphism of $E$ is the 
determinant of $\alpha$ acting on $\Lambda$. More generally, for an elliptic curve $E$ over an algebraically closed
field $\mathbf{k}$, and $l$ be a prime not equal to the characteristic of $\mathbf{k}$, 
\begin{equation*}
    {\rm deg}(\alpha)={\rm det}(\alpha|T_lE)
\end{equation*} 
where $T_lE$ is the Tate module $T_lE=\lim_{\longleftarrow} E(\mathbf{k})_{l^n} $ of $E$.
 \begin{proposition}
Let $E$ be an elliptic curve over $\mathbb{F}_p$. Then the trace of the Frobenius endomorphism $\varphi_p$ on $T_l E$ has the following formula
\begin{equation*}
    {\rm Tr}(\varphi_p|T_l E)= a_p:=p+1-N_p.
\end{equation*} 
\label{pro:def_TatesM}
 \end{proposition}

 \begin{corollary}
    Let $E$ be an elliptic curve over $\mathbb{F}_p$. Then 
    \begin{equation*}
        {\rm Tr} (\varphi_p^{tr}|T_l E)={\rm Tr}(\varphi_p|T_l E).
    \end{equation*}
    \label{cor:trace_transpose}
\end{corollary}
 \subsection{The action of the Hecke operators on $H_1(E,\mathbb{Z})$}
 \begin{corollary}
    For any $p\nmid N$,
    ${\rm Tr}(T(p)|H_1(X_0(N),\mathbb{Z}))={\rm Tr }(T(p)|\Omega^1(X_0(N)))+\overline{{\rm Tr}(T(p)|\Omega^1(X_0(N)))}$
    \label{cor:Tp_on_homology_group}
 \end{corollary}
 \begin{proof}
    The proof of above Proposition and Corollary is straightforward calculation on \cite{Mil06}, which we will skip. 
 \end{proof}

 \begin{theorem}
    Consider an $f=\sum c(n)q^n$ and a map $X_0(N)\rightarrow E$, as in Theorem \ref{Thm:Modularity}. For all $p\nmid N$,
    \begin{equation*}
        c(p)=a_p:=p+1-N_p
    \end{equation*}
 \end{theorem}
 \begin{proof}
    We assume initially that $X_0(N)$ has genus $1$. Then $X_0(N) \longrightarrow E$ is an isogeny, and we can take 
    $E=X_0(N)$. Let $p$ be a prime not dividing $N$. Then $E$ has good reduction at $p$, and for any $l\neq p$, the 
    reduction map $T_lE \longrightarrow T_l\widetilde{E}$ is an isomorphism. The Eichler-Shimura Relation states that 
    \begin{equation*}
        \widetilde{T}(p)=\varphi_p+\varphi_p^{\rm tr}
    \end{equation*}
    We take trace on $T_l\widetilde{E}$. According to the Proposition \ref{pro:def_TatesM}, Corollary \ref{cor:trace_transpose} and 
    Corollary \ref{cor:Tp_on_homology_group}, we get the following result
    \begin{equation*}
        c(p)=a_p:=p+1-N_p
    \end{equation*}
 \end{proof}

 \section{Another Proof of Eichler-Shimura Relation}
We will proof the Eichler-Shimura Relation by using the commutative diagram, this proof is mainly come from \cite{Fry05}.
\subsection{Further discussion for Hecke operator}
As we discuss above, there is a natural map between moduli space and modular curve,
\begin{equation*}
    \psi_1: {\rm S}_1(N) \longrightarrow X_1(N), \qquad [\mathbb{C}/\Lambda_{\tau} , 1/N +\Lambda_{\tau}]\mapsto \Gamma_1(N)\tau,
\end{equation*}
to divisor groups. To make $\psi_1$ algebraic, consider the following commutative diagram
\begin{equation*}
    \begin{CD}
        {\rm S}_1(N) @>>> {\rm S}_1(1) \\
        @V\psi_1VV @VVV\\
        X_1(N) @>>> X_1(N)
    \end{CD}
\end{equation*}
given by 
\begin{equation*}
    \begin{CD}
        [\mathbb{C}/\Lambda_{\tau} , 1/N +\Lambda_{\tau}] @>>> [\mathbb{C}/\Lambda_{\tau}]\\
        @VVV @VVV\\
        \Gamma_1(N)\tau @>>> {\rm SL}_2(\mathbb{Z})\tau 
    \end{CD}
\end{equation*}
As we discuss in Hecke operator, we have the following commutative diagram
\begin{equation*}
    \begin{CD}
        {\rm Div}({\rm S}_1 (N)) @>T_p>> {\rm Div}({\rm S}_1(N)) \\
        @V\psi_1VV @VV\psi_1V \\
        {\rm Div(X_1(N))} @>T_p>> {\rm Div}(X_1(N))
    \end{CD}
\end{equation*}
The map $\psi_1: {\rm S}_1(N) \longrightarrow Y_1(N)$ from the discussion on moduli space and modular curve. Now we restrict
the vertical maps to degree-zero divisors and then the bottom row passes to Picard groups, giving a commutative diagram
\begin{equation}
    \begin{CD}
        {\rm Div}^0({\rm S}_1 (N)) @>T_p>> {\rm Div}^0({\rm S}_1(N)) \\
        @V\psi_1VV @VV\psi_1V \\
        {\rm Pic}^0(X_1(N)) @>T_p>> {\rm Pic}^0(X_1(N))
    \end{CD}
    \label{eq:hecke_div_pic}
\end{equation}

\subsection{Proof of Eichler-Shimura Relation}

Let $N$ be a positive integer and let $p\nmid N$ be prime. We will first give a description of the Hecke 
operator $T_p$ at the level of Picard groups of reduced modular curves,
\begin{equation*}
    \widetilde{T_p}:{\rm Pic}^0(\widetilde{X}_1(N)) \longrightarrow {\rm Pic}^0(\widetilde{X}_1(N)).
\end{equation*}
By Theorem \ref{theorem:reduce_Picard} the Hecke operator $\left\langle d \right\rangle $ on $X_1(N)$
reduces modulo $p$ and passes to Picard groups to give a commutative diagram 
\begin{equation}
    \begin{CD}
        {\rm Pic}^0(X_1(N)) @>\left\langle d\right\rangle _*>> {\rm Pic}^0(X_1(N)) \\
        @VVV @VVV \\
        {\rm Pic}^0(\widetilde{X}_1(N)) @>\widetilde{\left\langle d \right\rangle}_*>> {\rm Pic}^0(\widetilde{X}_1(N)) 
    \end{CD}
    \label{eq:reduced_Picard}
\end{equation}
We want a similar diagram for the Hecke operator $T_p$ on ${\rm Pic}^0(X_1(N))$, 
\begin{equation}
    \begin{CD}
        {\rm Pic}^0(X_1(N)) @>T_p>> {\rm Pic}^0(X_1(N)) \\
        @VVV @VVV \\
        {\rm Pic}^0(\widetilde{X}_1(N)) @>\widetilde{T_p}>> {\rm Pic}^0(\widetilde{X}_1(N)) 
    \end{CD}
    \label{eq:reduced_Picard2}
\end{equation}
However, since the top row here is not the pushforward of a morphism from $X_1(N)$ to $X_1(N)$, Theorem \ref{theorem:reduce_Picard}
doesn't give this diagram as it gives Equation \ref{eq:reduced_Picard}.
The way to compute the reduction $\widetilde{T_p}$ on ${\rm Pic}^0(\widetilde{X}_1(N))$ is to compute it first
in the moduli space environment. We will establish a moduli sapce diagram analogous to Equation \ref{eq:reduced_Picard2} and then use 
Igusa's Theorem to translate it to modular curves. Recall the moduli space interpretation of the Hecke operator $T_p$
on the divisor group of ${\rm S}_1(N)$,
\begin{equation}
    T_p[E,Q]=\sum_C[E/C,Q+C].
    \label{eq:T_p}
\end{equation}
The sum is taken over all order $p$ subgroups $C\subset E$ such that $C\cap \left\langle Q\right\rangle =\{0\}$,
in this case all order $p$ subgroups since $p\nmid N$.\par
Let $E$ be an elliptic curve over $\overline{\mathbb{Q}}$ with ordinary reduction at $\mathfrak{p}$ and let 
$Q\in E$ be a point of order $N$. Let $C_0$ be the kernel of the reduction map $E[p]\longrightarrow \widetilde{E}[p]$,
an order $p$ subgroup of $E$ since the ma surjects and the reduction is ordinary. 
\begin{lemma}
    For any order $p$ subgroup $C$ of $E$, 
    \begin{equation*}
        [\widetilde{E/C},\widetilde{Q+C}]=\begin{cases}
            [\widetilde{E}^{\varphi_p},\widetilde{Q}^{\varphi_p}] & \text{if } C=C_0\\
            [\widetilde{E}^{\varphi_p^{-1}},[p]\widetilde{Q}^{\varphi_p^{-1}}] & \text{if } C\neq C_0
        \end{cases}
    \end{equation*}
    \label{lm}
\end{lemma}
There are $p+1$ order $p$ subgroups $C$ of $E$, one of which is $C_0$. Define the moduli space diamond 
operator in characteristic $p$ to be 
\begin{equation*}
    \widetilde{\left\langle d\right\rangle }:\widetilde{{\rm S}}_1(N) \longrightarrow \widetilde{{\rm S}}_1(N), \qquad [E,Q]\mapsto [E,[d]Q], \quad (d,N)=1.
\end{equation*}
Then summing the formula of above Lemma over all order $p$ subgroups $C\subset E$ gives for a curve $E$ with ordinary
reduction at $\mathfrak{p}$, 
\begin{equation}
    \sum_C [\widetilde{E/C}, \widetilde{Q+C}]=(\varphi_p+p\widetilde{\left\langle p\right\rangle }\varphi_p^{-1})[\widetilde{E},\widetilde{Q}].
    \label{eq:reduced_T_p}
\end{equation}

The above Lemma extends to supersingular curves. If $E$ is an elliptic curve over $\overline{\mathbb{Q}}$ with supersingular
reduction at $\mathfrak{p}$ and $Q\in E$ is a point of order $N$, then for any order $p$ subgroup $C$ of $E$ 
\begin{equation*}
    [\widetilde{E/C},\widetilde{Q+C}]=[\widetilde{E}^{\varphi_p}, \widetilde{Q}^{\varphi_p}]=[\widetilde{E}^{\varphi_p^{-1}},[p]\widetilde{Q}^{\varphi_p^{-1}}].
\end{equation*}
Summing this over the $p+1$ such subgroups $C$ of $E$, using the first expression for $[\widetilde{E/C},\widetilde{Q+C}]$ 
once and the second expression $p$ times, shows that formula \ref{eq:reduced_T_p} applies to curves with supersingular
reduction at $\mathfrak{p}$ as well. Therefore it applies to all curves with good reduction at $\mathfrak{p}$.\par
If an elliptic curve $\widetilde{E}$ over $\overline{\mathbb{F}}_p$ has invariant $j\notin \{0,1728\}$ then the same
holds for $\widetilde{E}^{\varphi_p}$ and $\widetilde{E}^{\varphi_p^{-1}}$. Formulas \ref{eq:T_p} and \ref{eq:reduced_T_p}
combine with this observation to give a commutative diagram, where as before the primes mean to avoid some points,
only finitely many in characteristic $p$,
\begin{equation*}
    \begin{CD}
        {\rm S}_1(N)'_{\rm gd} @>T_p>> {\rm Div}({\rm S}_1(N)'_{\rm gd}) \\
        @VVV @VVV\\
        \widetilde{{\rm S}}_1(N)' @>\varphi_p+p\widetilde{\left\langle p \right\rangle }\varphi_p^{-1}>> {\rm Div}(\widetilde{{\rm S}}_1(N)')
    \end{CD}
\end{equation*}
That is, $T_p$ on ${\rm S}_1(N)'_{\rm gd}$ reduces at $\mathfrak{p}$ to $\varphi_p+p\widetilde{\left\langle p \right\rangle }\varphi_p^{-1}$.
This extends to divisor groups and then restricts to degree-0 divisors, 
\begin{equation}
    \begin{CD}
        {\rm Div}^0({\rm S}_1(N)'_{\rm gd}) @>T_p>> {\rm Div}({\rm S}_1(N)'_{\rm gd}) \\
        @VVV @VVV\\
        {\rm Div}^0(\widetilde{{\rm S}}_1(N)') @>\varphi_p+p\widetilde{\left\langle p \right\rangle }\varphi_p^{-1}>> {\rm Div}(\widetilde{{\rm S}}_1(N)')
    \end{CD}
    \label{cd:T_p_lemma}
\end{equation}
We also have a commutative diagram 
\begin{equation}
    \begin{CD}
        {\rm Div}^0(\widetilde{{\rm S}}_1(N)') @>\varphi_p+p\widetilde{\left\langle p \right\rangle }\varphi_p^{-1}>> {\rm Div}(\widetilde{{\rm S}}_1(N)') \\
        @VVV @VVV \\
        {\rm Pic}^0(\widetilde{X}_1(N)) @>\varphi_{p,*}+\widetilde{\left\langle p \right\rangle }_*\varphi_p^*>> {\rm Pic}^0(\widetilde{X}_1(N))
    \end{CD}
    \label{cd:lemma}
\end{equation}
A cube-shaped diagram now exist as follows, where all squares except possibly the back one commute.

\begin{center}
    \begin{tikzcd}[row sep=scriptsize, column sep=scriptsize]
        & {\rm Pic}^0(X_1(N)) \arrow[rr,"T_p"] \arrow[dd] & & {\rm Pic}^0(X_1(N))  \arrow[dd] \\
         {\rm Div}^0({\rm S}_1(N)'_{\rm gd}) \arrow[ur] \arrow[rr, crossing over, "T_p" near start] \arrow[dd] & & {\rm Div}^0({\rm S}_1(N)'_{\rm gd}) \arrow[ur] \\
        & {\rm Pic}^0(\widetilde{X}_1(N))  \arrow[rr,"\varphi_{p,*}+\widetilde{\left\langle p \right\rangle }_*\varphi_p^* " near start] & & {\rm Pic}^0(\widetilde{X}_1(N))  \\
        {\rm Div}^0({\rm \widetilde{S}}_1(N)') \arrow[rr,"\varphi_p+p\widetilde{\left\langle p \right\rangle }\varphi_p^{-1}"] \arrow[ur] & & {\rm Div}^0({\rm \widetilde{S}}_1(N)') \arrow[from=uu, crossing over] \arrow[ur] \\
        \end{tikzcd}
\end{center}
To establish this, note that \par
(1) The top square, a diagram in characteristic 0 relating $T_p$ on the moduli space and on the modular curve, is a 
restriction of commutative diagram \ref{eq:hecke_div_pic} \par
(2) The side squares, relating reduction at $p$ to the map from the moduli space to the modular curve, are commutative diagram
from \ref{eq:Igusa_theorem} Igusa's Theorem.\par
(3) The front square, relating $T_p$ on the moduli space to reduction at $p$, is commutative diagram \ref{cd:T_p_lemma} resulting from
Lemma \ref{lm}. \par
(4) The bottom square, relating maps in characteristic $p$ on the moduli space and on the modular curve, is commutative diagram \ref{cd:lemma}.\par
Therefore, the back square is commute. Moreover, the cube commute. We get the following result
\begin{theorem}[Eichler-Shimura Relation]
    Let $p\nmid N$. The following diagram commutes:
    \begin{equation*}
        \begin{CD}
            {\rm Pic}^0(X_1(N)) @>T_p>> {\rm Pic}^0(X_1(N)) \\
            @VVV @VVV \\
            {\rm Pic}^0(\widetilde{X}_1(N))  @>\varphi_{p,*}+\widetilde{\left\langle p \right\rangle }_*\varphi_p^*>> {\rm Pic}^0(\widetilde{X}_1(N))  \\
        \end{CD}
    \end{equation*}
    In particular since $\widetilde{\left\langle p\right\rangle } $ acts trivially on $\widetilde{X}_0(N)$, the following
    diagram commutes as well:
    \begin{equation*}
        \begin{CD}
            {\rm Pic}^0(X_0(N)) @>T_p>> {\rm Pic}^0(X_0(N)) \\
            @VVV @VVV \\
            {\rm Pic}^0(\widetilde{X}_0(N))  @>\varphi_{p,*}+\varphi_p^*>> {\rm Pic}^0(\widetilde{X}_0(N))  \\
        \end{CD}
    \end{equation*}
\end{theorem}




We main use the book \cite{Mil06} and \cite{Fry05} and \cite{Atk70}
\clearpage

 \printbibliography[title={Bibliography}]
\end{document}